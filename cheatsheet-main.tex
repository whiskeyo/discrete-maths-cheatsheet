\documentclass[10pt,landscape,a4paper]{article}
\usepackage[utf8]{inputenc}
\usepackage[ngerman]{babel}
\usepackage[T1]{fontenc}
%\usepackage[LY1,T1]{fontenc}
%\usepackage{frutigernext}
%\usepackage[lf,minionint]{MinionPro}
\usepackage{tikz}
\usetikzlibrary{shapes,positioning,arrows,fit,calc,graphs,graphs.standard}
\usepackage[nosf]{kpfonts}
\usepackage[t1]{sourcesanspro}
\usepackage{multicol}
\usepackage{wrapfig}
\usepackage[top=2mm,bottom=5mm,left=2mm,right=2mm]{geometry}
\usepackage[framemethod=tikz]{mdframed}
\usepackage{microtype}
\usepackage{pdfpages}

%% moje definicje:
\usepackage{enumerate}
\usepackage{amssymb}		% pakiet do symboli
\usepackage{mathtools}		% pakiet do matmy (rozszerza amsmath)
\usepackage{enumitem}		% punktowanie (a), (b), ...
\usepackage{nopageno}		% brak numerow stron
\usepackage{graphicx}		% wstawianie obrazkow
\usepackage{float}			% wstawianie obrazkow w dowolnym miejscu
\usepackage{caption}
\usepackage{esdiff}         % pochodne \diff{}{}
\usepackage{listings}
\usepackage{xcolor}
\usepackage{adjustbox}
\usepackage[none]{hyphenat} % usunięcie łamania wyrazów na końcu linii

% nowe komendy dla wygodniejszego pisania :)

\newcommand{\floor}[1]{\left\lfloor #1 \right\rfloor}	% podłoga
\newcommand{\ceil}[1]{\left\lceil #1 \right\rceil}		% sufit
\newcommand{\fractional}[1]{\left\{ #1 \right\}}		% część ułamkowa {x}
\newcommand{\abs}[1]{\left| #1 \right|}					% wartosc bezwzgledna / moc
\newcommand{\set}[1]{\left \{ #1 \right \}}				% zbiór elementów {a,b,c}
\newcommand{\pair}[1]{\left( #1 \right)}				% para elementów (a,b)
\newcommand{\Mod}[1]{\ \mathrm{mod\ #1}}				% lekko zmodyfikowane modulo
\newcommand{\conv}[1]{\equiv_{#1}}                      % przystawanie
\newcommand{\comp}[1]{\overline{ #1 }} 					% dopełnienie zbioru 
\newcommand{\annihilator}{\mathbf{E}}					% operator E
\newcommand{\seqAnnihilator}[1]{\annihilator \left\langle #1 \right\rangle} % E(a_n)
\newcommand{\sequence}[1]{\left\langle #1 \right\rangle} % <a_n>
\DeclareMathOperator{\lcm}{lcm}							% obsługa lcm w mathmode

% styl do kodu
\lstdefinestyle{code}{%
basicstyle=\ttfamily\small,
commentstyle=\color{green!60!black},
keywordstyle=\color{magenta},
stringstyle=\color{blue!50!red},
showstringspaces=false,
%numbers=left,
%numberstyle=\footnotesize\color{gray},
%numbersep=10pt,
tabsize=4,
rulecolor=\color{red},
breaklines=true
}

\newcommand{\code}[1]{\lstinline[style=code]{#1}} % kod inline
%%

\let\bar\overline

\definecolor{myblue}{cmyk}{1,.72,0,.38}

\def\firstcircle{(0,0) circle (1.5cm)}
\def\secondcircle{(0:2cm) circle (1.5cm)}

\colorlet{circle edge}{myblue}
\colorlet{circle area}{myblue!5}

\tikzset{filled/.style={fill=circle area, draw=circle edge, thick},
    outline/.style={draw=circle edge, thick}}
    
\pgfdeclarelayer{background}
\pgfsetlayers{background,main}

\everymath\expandafter{\the\everymath \color{myblue}}
\everydisplay\expandafter{\the\everydisplay \color{myblue}}

\renewcommand{\baselinestretch}{.8}
\pagestyle{empty}

\global\mdfdefinestyle{header}{%
linecolor=gray,linewidth=1pt,%
leftmargin=0mm,rightmargin=0mm,skipbelow=0mm,skipabove=0mm,
}

\newcommand{\header}{
\begin{mdframed}[style=header]
\footnotesize
\sffamily
Matematyka dyskretna (L) - cheatsheet\\
Tomasz Woszczyński
\end{mdframed}
}

\makeatletter % Author: https://tex.stackexchange.com/questions/218587/how-to-set-one-header-for-each-page-using-multicols
\renewcommand{\section}{\@startsection{section}{1}{0mm}%
                                {.2ex}%
                                {.2ex}%x
                                {\color{myblue}\sffamily\small\bfseries}}
\renewcommand{\subsection}{\@startsection{subsection}{1}{0mm}%
                                {.2ex}%
                                {.2ex}%x
                                {\sffamily\bfseries}}
\renewcommand{\subsubsection}{\@startsection{subsubsection}{1}{0mm}%
                                {.2ex}%
                                {.2ex}%x
                                {\sffamily\bfseries}}


\def\multi@column@out{%
   \ifnum\outputpenalty <-\@M
   \speci@ls \else
   \ifvoid\colbreak@box\else
     \mult@info\@ne{Re-adding forced
               break(s) for splitting}%
     \setbox\@cclv\vbox{%
        \unvbox\colbreak@box
        \penalty-\@Mv\unvbox\@cclv}%
   \fi
   \splittopskip\topskip
   \splitmaxdepth\maxdepth
   \dimen@\@colroom
   \divide\skip\footins\col@number
   \ifvoid\footins \else
      \leave@mult@footins
   \fi
   \let\ifshr@kingsaved\ifshr@king
   \ifvbox \@kludgeins
     \advance \dimen@ -\ht\@kludgeins
     \ifdim \wd\@kludgeins>\z@
        \shr@nkingtrue
     \fi
   \fi
   \process@cols\mult@gfirstbox{%
%%%%% START CHANGE
\ifnum\count@=\numexpr\mult@rightbox+2\relax
          \setbox\count@\vsplit\@cclv to \dimexpr \dimen@-1cm\relax
\setbox\count@\vbox to \dimen@{\vbox to 1cm{\header}\unvbox\count@\vss}%
\else
      \setbox\count@\vsplit\@cclv to \dimen@
\fi
%%%%% END CHANGE
            \set@keptmarks
            \setbox\count@
                 \vbox to\dimen@
                  {\unvbox\count@
                   \remove@discardable@items
                   \ifshr@nking\vfill\fi}%
           }%
   \setbox\mult@rightbox
       \vsplit\@cclv to\dimen@
   \set@keptmarks
   \setbox\mult@rightbox\vbox to\dimen@
          {\unvbox\mult@rightbox
           \remove@discardable@items
           \ifshr@nking\vfill\fi}%
   \let\ifshr@king\ifshr@kingsaved
   \ifvoid\@cclv \else
       \unvbox\@cclv
       \ifnum\outputpenalty=\@M
       \else
          \penalty\outputpenalty
       \fi
       \ifvoid\footins\else
         \PackageWarning{multicol}%
          {I moved some lines to
           the next page.\MessageBreak
           Footnotes on page
           \thepage\space might be wrong}%
       \fi
       \ifnum \c@tracingmulticols>\thr@@
                    \hrule\allowbreak \fi
   \fi
   \ifx\@empty\kept@firstmark
      \let\firstmark\kept@topmark
      \let\botmark\kept@topmark
   \else
      \let\firstmark\kept@firstmark
      \let\botmark\kept@botmark
   \fi
   \let\topmark\kept@topmark
   \mult@info\tw@
        {Use kept top mark:\MessageBreak
          \meaning\kept@topmark
         \MessageBreak
         Use kept first mark:\MessageBreak
          \meaning\kept@firstmark
        \MessageBreak
         Use kept bot mark:\MessageBreak
          \meaning\kept@botmark
        \MessageBreak
         Produce first mark:\MessageBreak
          \meaning\firstmark
        \MessageBreak
        Produce bot mark:\MessageBreak
          \meaning\botmark
         \@gobbletwo}%
   \setbox\@cclv\vbox{\unvbox\partial@page
                      \page@sofar}%
   \@makecol\@outputpage
     \global\let\kept@topmark\botmark
     \global\let\kept@firstmark\@empty
     \global\let\kept@botmark\@empty
     \mult@info\tw@
        {(Re)Init top mark:\MessageBreak
         \meaning\kept@topmark
         \@gobbletwo}%
   \global\@colroom\@colht
   \global \@mparbottom \z@
   \process@deferreds
   \@whilesw\if@fcolmade\fi{\@outputpage
      \global\@colroom\@colht
      \process@deferreds}%
   \mult@info\@ne
     {Colroom:\MessageBreak
      \the\@colht\space
              after float space removed
              = \the\@colroom \@gobble}%
    \set@mult@vsize \global
  \fi}

\makeatother
\setlength{\parindent}{0pt}

\begin{document}
%\footnotesize
\small
\begin{multicols*}{5}
\section{Wariacje}
\subsection*{Liczba wariacji z powtórzeniami}
Dla zbiorów $A, B$ o odpowiednio $m, n$ elementach liczba funkcji ze zbioru $A$ w $B$
wynosi $n^m$, czyli $\abs{\{ f: A \to B \} = n^m}$.

\subsection*{Liczba wariacji bez powtórzeń}
Dla zbiorów $A, B$ o odpowiednio $m, n$ elementach liczba funkcji różnowartościowych
ze zbioru $A$ w $B$ wynosi $n (n - 1) \ldots (n - m + 1) = \frac{n!}{(n - m)!}$.

\subsection*{Liczba podzbiorów}
Zbiór $A$ o $n$ elementach ma $\abs{\{ B : B \subseteq A \} = 2^n}$ podzbiorów.

\subsection*{Para podzbiorów}
Dla $U$ będącego $n$-elementowym można wyznaczyć dwa jego podzbiory $A, B$ takie,
że $A \subseteq B$ na $\abs{\{ (A,B) : A \subseteq B \subseteq U \}} = 
\abs{\{ f: U \to \{ 0, 1, 2 \} \}} = 3^n$ sposobów.

\subsection*{Liczba permutacji}
Zbiór $U$ o $n$ elementach można spermutować na $n!$ sposobów.

\subsection*{Sufit, podłoga, część ułamkowa}
Niech $x \in \mathbb{R}, n \in \mathbb{Z}$, wtedy: \\
$ \floor{x} = n \Leftrightarrow n \leq x < n + 1 $ \\
$ \ceil{x}  = n \Leftrightarrow n - 1 < x \leq n $ \\
$ \fractional{x} = x - \floor{x} $

\subsection*{Własności sufitu i podłogi}
Niech $x \in \mathbb{R}, n \in \mathbb{Z}$, wtedy: \\
$ \floor{x + n} = n + \floor{x} $, ponieważ \\
$ \floor{x} + n \leq x + n < \floor{x} + n + 1 $. \\
Ponadto mamy: \\
$ \ceil{x + n} = n + \ceil{x} $ \\
$ \floor{-x} = -\ceil{x} $

\subsection*{Podzbiory k-elementowe}
Niech $\abs{U} = \{ 1, 2, \ldots, n \}$ oraz $P_n^k = {\{ A \subseteq U : \abs{A} = k \}}$.
Wtedy ${\frac{n!}{(n - k)!}} = {k! \abs{P_n^k}}$, czyli 
$\abs{P_n^k} = \frac{n!}{(n - k)! k!} = \binom{n}{k}$.

\subsection*{Symbol Newtona}
Dla $k, n \in \mathbb{N}$ takich, że $0 \leq k \leq n$ zachodzi:
$\binom{n}{k} = \binom{n}{n-k}$ \\
$\binom{n}{k} + \binom{n}{k+1} = \binom{n+1}{k+1}$

\subsection*{Kulki i szufladki}
$n$ kulek do $k$ szuflad można wrzucić na tyle sposobów, ile jest ciągów złożonych
z $n$ zer i $k-1$ jedynek, czyli $\binom{n+k-1}{k-1}$.

\subsection*{Dwumian Newtona}
Dla $n \in \mathbb{N}$ mamy 
$(x+y)^n = \sum\limits_{i=0}^{n} \binom{n}{i} x^i y^{n-i}$.

\subsection*{Zasada szufladkowa Dirichleta}
Niech $k, s \in \mathbb{N}_+$. Jeśli wrzucimy $k$ kulek do $s$ szuflad (Dirichleta),
a kulek jest więcej niż szuflad ($k > s$), to w którejś szufladzie będą przynajmniej
dwie kulki. Innymi słowy, dla skończonych zbiorów $A, B$, jeśli $\abs{A} > \abs{B}$,
to nie istnieje funkcja różnowartościowa z $A$ w $B$. Dla $k > s\cdot i$ kulek oraz
$s$ szuflad będzie w jakiejś szufladzie $i + 1$ kulek.
\section{Asymptotyka}
Niech $f, g : \mathbb{N} \to \mathbb{R} \geq 0$, wtedy możemy mówić o takich
funkcjach asymptotycznych:

\subsection*{Notacja dużego $O$}
Mamy ${f(n) = O(g(n))}$ wtw, gdy
${\exists (c > 0)} \ {\exists (n_0 \in \mathbb{N})} \ {\forall (n \geq n_0)} \
{f(n) < cg(n)}$. Ponadto dla ${C, a, \alpha, \beta \in \mathbb{R}}$ zachodzą takie
własności: \\
1. $\forall (\alpha, \beta) \ \alpha \leq \beta \Rightarrow n^\alpha = O(n^\beta)$, \\
2. $\forall (\alpha > 1) \ n^C = O(a^n)$, \\
3. $\forall (\alpha > 0) \ (\ln n)^C = O(n^\alpha)$. \\
Przydatna może okazać się reguła de l'Hospitala, więc gdy $f(n)$ i $g(n)$ dążą do
nieskończoności, to ${\lim\limits_{n \to \infty} \frac{f(n)}{g(n)}} = 
{\lim\limits_{n \to \infty} \frac{f^\prime (n)}{g^\prime (n)}}$.

\subsection*{Notacja małego $o$}
${f(n) = o(g(n))}$ wtw, gdy $\lim\limits_{n \to \infty} \frac{f(n)}{g(n)} = 0$.

\subsection*{Notacja duże Omega ($\Omega$)}
$f(n) = \Omega(g(n))$ wtw, gdy ${\exists (c > 0)} {\exists (n_0 \in \mathbb{N})} 
{\forall (n \geq n_0)}  {f(n) \geq cg(n)}$.

\subsection*{Notacja Theta ($\Theta$)}
$f(n) = \Theta(g(n))$ wtw, gdy ${f(n) = \Omega(g(n))} \wedge {f(n) = O(g(n))}$.

\subsection*{Notacja małe Omega ($\omega$)}
$f(n) = \omega(g(n))$ wtw, gdy $\lim\limits_{\mathclap{n \to \infty}} = \frac{f(n)}{g(n)} = \infty$.
\section{Arytmetyka modularna}

\subsection*{Funkcja modulo}
Niech $n, d \in \mathbb{Z}$ i $d \neq 0$. Wtedy: \\
$n \Mod{d} = n - \floor{\frac{n}{d}} \cdot d$. \\

$n \Mod{d} = r$ wtw, gdy ${0 \leq r < d} \ \wedge \ {\exists (k \in \mathbb{Z})} \ {n = kd + r}$

\subsection*{Przystawanie modulo}
$a \conv{n} b$ wtw, gdy $a \Mod{n} = b \Mod{n}$

\subsection*{Własności funkcji modulo}
$a + b \conv{n} a \Mod{n} + b \Mod{n}$ \\
$a \cdot b \conv{n} (a \Mod{n}) \cdot (b \Mod{n})$

\subsection*{Podzielność}
Niech $n, d \in \mathbb{Z}$ i $d \neq 0$. Wtedy: \\
$d | n$ wtw, gdy $\exists (k \in \mathbb{Z}) \ n = kd$ \\
$d | n$ wtw, gdy $n \Mod{d} = 0$ \\
$d | n$ wtw, gdy $n \conv{d} 0$ \\
$d | n_1 \wedge d | n_2$ to $d | (n_1 + n_2)$

\subsection*{Największy wspólny dzielnik (NWD, gcd)}
Niech $a, b \in \mathbb{N}$, wtedy \\
$\gcd(a,b) = \max \{ d \in \mathbb{N} : d|a \wedge d|b \}$

\subsection*{Algorytm Euklidesa}
Dla $a \geq b > 0$ korzystamy z własności:
$\gcd(a,b) = \gcd(b, a \Mod{b})$ oraz $\gcd(a, 0) = a$.
\begin{lstlisting}[style=code]
gcd(a, b):
    while b != 0:
        c = a mod b
        a = b
        b = c
    return a
\end{lstlisting}

\subsection*{Rozszerzony algorytm Euklidesa}
Dla $a \geq b > 0$: \\
$\exists (x, y \in \mathbb{Z}) \ xa + yb = \gcd(a, b)$
\begin{lstlisting}[style=code]
gcd(a, b):
    x = 1, y = 0, r = 0, s = 1
    while b != 0:
        c = a mod b
        q = a div b
        a = b
        b = c

        r' = r
        s' = s
        r = x - q * r
        s = y - q * s
        x = r'
        y = s'
    
    return a, x, y
\end{lstlisting}

\subsection*{Liczby względnie pierwsze}
Niech $a, b \in \mathbb{Z}$, wtedy te liczby są względnie pierwsze, gdy
$\gcd(a, b) = 1$.
\section{Wzór włączeń i wyłączeń}

$\abs{\bigcup\limits_{i=1}^{n} A_i} = 
\sum\limits_{k=1}^{n} (-1)^{k-1}
\sum\limits_{I \subseteq \{ 1,\ldots,n \}} \abs{\bigcap\limits_{i \in I} A_i}$
\section{Rekurencja, zależności rekurencyjne}

\subsection*{Liczby Fibonacciego}
Niech $F_0 = 0, F_1 = 1$, wtedy $F_n = F_{n-1} + F_{n-2}$ dla $n > 1$. 

\subsection*{Własności liczb Fibonacciego}
Każde dwie kolejne liczby Fibonacciego są względnie pierwsze. \\
$\gcd(F_m, F_n) = F_{\gcd(m, n)}$

\subsection*{Szereg harmoniczna}
$H_n = H_{n-1} + \frac{1}{n}$

\subsection*{Podział płaszczyzny na obszary}
$$
p_n =
\begin{cases}
    1 &\text{ dla } n = 0 \\
    p_{n-1} + n &\text{ dla } n \geq 1 
\end{cases}
$$

\subsection*{Liczba nieporządków $n$-elementowych}
1. $d_n = n! \cdot \sum\limits_{i=0}^{n} \frac{(-1)^i}{i!}$ \\
2. $d_{n+1} = n(d_n + d_{n+1})$ dla $d_0 = 1, d_1 = 0$.

\subsection*{Operator przesunięcia $\annihilator$}
Mamy ciąg $\sequence{a_n} = \sequence{a_0, a_1, \ldots, a_n, \ldots}$. Wtedy
$\seqAnnihilator{a_n} = \sequence{a_{n+1}} = \sequence{a_1, \ldots, a_n, \ldots}$.

\subsection*{Złożenie operatora przesunięcia}
$\annihilator^2 \sequence{a_n} = \annihilator \left( \seqAnnihilator{a_n} \right)
= \sequence{a_2, \ldots, a_n, \ldots}$

\subsection*{Operatory działające na ciągi}
1. $\sequence{a_n} + \sequence{b_n} = \sequence{a_n + b_n} = \sequence{a_0 + b_0, 
\ldots}$ \\
2. $c \sequence{a_n} = \sequence{c a_n} = \sequence{ca_0, ca_1, \ldots}$

\subsection*{Co anihiluje dane ciągi?}
1. $\sequence{\alpha} \Longrightarrow \annihilator - 1$. \\
2. $\sequence{\alpha a^i} \Longrightarrow \annihilator - a$. \\
3. $\sequence{\alpha a^i + \beta b^i} \Longrightarrow (\annihilator - a)(\annihilator - b)$. \\
4. $\sequence{\sum\limits_{k=0}^{n} \alpha_k a_k^i} \Longrightarrow  
\prod\limits_{k=0}^{n} (\annihilator - a_k)$. \\
5. $\sequence{\alpha i + \beta} \Longrightarrow (\annihilator - 1)^2$. \\
6. $\sequence{(\alpha i + \beta) a^i} \Longrightarrow (\annihilator - a)^2$. \\
7. $\sequence{(\alpha i + \beta) a_i + \gamma b^i} \Longrightarrow 
(\annihilator - a)^2 (\annihilator - b)$. \\
8. $\sequence{\sum\limits_{k=0}^{n-1} \alpha_k i^k} a^i \Longrightarrow 
(\annihilator - a)^n$. \\

\subsection*{Dodatkowe własności anihilatorów}
Jeśli $\annihilator_A$ anihiluje $\sequence{a_i}$, to ten sam anihilator anihiluje
również ciąg $c \sequence{a_n}$ dla dowolnej stałej $c$. \\
Jeśli $\annihilator_A$ anihiluje $\sequence{a_i}$ i $\annihilator_B$ anihiluje
$\sequence{b_i}$, to $\annihilator_A \annihilator_B$ anihiluje
$\sequence{a_i} \pm \sequence{b_i}$.

\subsection*{Liczby Catalana}
$C_n$ oznacza $n$-tą liczbę Catalana, wyraża się przez 
$C_n = \sum\limits_{i=1}^{n} C_{i-1} C_{n-i}$ dla $C_0 = 0$. Można je również
przedstawić wzorami $C_n = \frac{1}{n+1} \binom{2n}{n} = \frac{(2n)!}{(n+1)!n!}$.
Spełniają one zależność $C_n = \binom{2n}{n} - \binom{2n}{n+1}$. \\
Liczby Catalana posiadają różne interpretacje kombinatoryczne, takie jak
liczba poprawnych rozmieszczeń nawiasów, liczba dróg w układzie współrzędnych
w I ćwiartce, liczba drzew binarnych, liczba podziałów wielokąta wypukłego na trójkąty.

\subsection*{Funkcje tworzące (OGF)}
Dla ciągu $\sequence{a_n}$ można utworzyć funkcję 
$\sum\limits_{i=0}^{\infty} a_i x^i = a_0 + a_1 x + a_2 x^2 + \ldots = A(x)$, która
jest funkcją tworzącą tego ciągu. Poniżej kilka typowych funkcji tworzących dla
ciągów: \\
1. $\frac{1}{1-x}$ dla ciągu $\sequence{1}$, czyli $\frac{n}{1-x}$ dla $\sequence{n}$. \\
2. $\frac{1}{1-2x}$ dla ciągu $\sequence{2^n}$. \\
3. $\frac{1}{(1-x)^2}$ dla ciągu $\sequence{1, 2, 3, \ldots}$. \\
4. $\frac{1}{1-x^2}$ dla ciągu $\sequence{0, 1, 0, 1, \ldots}$. \\

\subsection*{Przesunięcie wyrazów w prawo o $k$ miejsc}
Aby z ciągu $\sequence{a_0, a_1, a_2, \ldots}$ o OGF $A(x)$ otrzymać ciąg 
$\sequence{0, \ldots, 0, a_0, a_1, \ldots}$, w którym pierwsze $k$ wyrazów jest $0$,
należy pomnożyć funkcję tworzącą przez $x^k$, więc mamy $x^k A(x)$.

\subsection*{Przesunięcie wyrazów w lewo o $k$ miejsc}
Aby z takiego ciągu jak wyżej otrzymać ciąg $\sequence{a_{k}, a_{k+1}, \ldots}$,
należy wykonać takie działanie: \\
$\frac{A(x) - \left( a_0 x^0 + a_1 x^1 + \ldots + a_{k-1} x^{k-1} \right)}{x^k}$.

\subsection*{Przerwy pomiędzy wyrazami}
Funkcją tworzącą takiego ciągu $\sequence{a_0, 0, a_1, 0, a_2, 0, \ldots}$ jest
$\sum\limits_{i=0}^{\infty} a_i x^i = a_0 + a_1 x^2 + a_2 x^4 + \ldots = A(x^2)$.
Dla ciągu o wyrazach co $3$ miejsca byłoby to $A(x^3)$, dla $4$ to $A(x^4)$, dla
$n$ więc $A(x^n)$.

\subsection*{Co drugi wyraz ciągu (pochodne)}
Funkcją tworzącą $\sequence{a_0, 0, a_2, 0, a_4, 0, \ldots}$ jest 
$\frac{A(x) + A(-x)}{2}$, dla $\sequence{0, a_1, 0, a_3, \ldots}$ mamy
$\frac{A(x) - A(-x)}{2}$. \\ \\
Funkcją tworzącą takiego ciągu 
$\sequence{0, a_1, 2a_2, 3a_3, 4a_4, \ldots, ia_i, \ldots}$
jest pochodna funkcji $A(x)$ przesunięta o jedno miejsce w prawo, a więc $xA^\prime(x)$.

\subsection*{Wykorzystanie całek w OGF}
Aby odnaleźć funkcję tworzącą ciągu 
$\sequence{0, \frac{a_1}{1}, \frac{a_2}{2}, \ldots, \frac{a_i}{i}, \ldots}$ należy
scałkować funkcję tworzącą $A(x)$ i przesunąć ją w lewo:
$\int\limits_{0}^{1} \frac{A(x) - a_0}{x} dx = 
\sum\limits_{i=0}^{\infty} \frac{a_i}{i} x^{i}$.

\subsection*{Inne funkcje tworzące}
1. $\sequence{n^2}$ odpowiada OGF $\frac{x(1+x)}{(1-x)^3}$. \\
2. $\sequence{n^3}$ odpowiada OGF $x \frac{x^2 + 4x + 1}{(1-x)^4}$. \\
3. $\sequence{\binom{n+k}{k}}$ odpowiada OGF $\frac{1}{(1-x)^{n+1}}$.

\subsection*{Liczba podziałów liczby $n$}
1. Dowolne składniki: $\prod\limits_{i=1}^{\infty} \frac{1}{1 - x^i}$ \\
2. Różne składniki: $\prod\limits_{i=1}^{\infty} (1 + x^i)$ \\
3. Nieparzyste składniki: $\prod\limits_{i=1}^{\infty} (1 + x^{2i-1})$ \\
4. Składniki mniejsze od $m$: $\prod\limits_{i=1}^{m-1} \frac{1}{1 - x^i}$ \\
5. Różne potęgi $2$: $\prod\limits_{i=1}^{\infty} (1 + x^{2^i})$

\subsection*{Rekursja uniwersalna}
Niech $a, b, c$ będą dodatnimi stałymi, rozwiązaniem równania rekurencyjnego
$$
T(n) =
\begin{cases}
    b                    &\text{ dla } n = 1\\
    aT(\frac{n}{c}) + bn &\text{ dla } n > 1
\end{cases}
$$
dla $n$ będących potęgą liczby $c$ jest
$$
T(n) =
\begin{cases}
    O(n)                         &\text{ dla } a < c \\
    O(n \log n)                  &\text{ dla } a = c \\
    O\left( n^{\log_c a} \right) &\text{ dla } a > c
\end{cases}
$$
\end{multicols*}
\end{document}