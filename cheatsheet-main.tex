\documentclass[10pt,landscape,a4paper]{article}
\usepackage[utf8]{inputenc}
\usepackage[ngerman]{babel}
\usepackage[T1]{fontenc}
%\usepackage[LY1,T1]{fontenc}
%\usepackage{frutigernext}
%\usepackage[lf,minionint]{MinionPro}
\usepackage{tikz}
\usetikzlibrary{shapes,positioning,arrows,fit,calc,graphs,graphs.standard}
\usepackage[nosf]{kpfonts}
\usepackage[t1]{sourcesanspro}
\usepackage{multicol}
\usepackage{wrapfig}
\usepackage[top=2mm,bottom=5mm,left=2mm,right=2mm]{geometry}
\usepackage[framemethod=tikz]{mdframed}
\usepackage{microtype}
\usepackage{pdfpages}

%% moje definicje:
\usepackage{enumerate}
\usepackage{amssymb}		% pakiet do symboli
\usepackage{mathtools}		% pakiet do matmy (rozszerza amsmath)
\usepackage{enumitem}		% punktowanie (a), (b), ...
\usepackage{nopageno}		% brak numerow stron
\usepackage{graphicx}		% wstawianie obrazkow
\usepackage{float}			% wstawianie obrazkow w dowolnym miejscu
\usepackage{caption}
\usepackage{esdiff}         % pochodne \diff{}{}
\usepackage{listings}
\usepackage{xcolor}
\usepackage{adjustbox}
\usepackage[none]{hyphenat} % usunięcie łamania wyrazów na końcu linii

% nowe komendy dla wygodniejszego pisania :)

\newcommand{\floor}[1]{\left\lfloor #1 \right\rfloor}	% podłoga
\newcommand{\ceil}[1]{\left\lceil #1 \right\rceil}		% sufit
\newcommand{\fractional}[1]{\left\{ #1 \right\}}		% część ułamkowa {x}
\newcommand{\abs}[1]{\left| #1 \right|}					% wartosc bezwzgledna / moc
\newcommand{\set}[1]{\left \{ #1 \right \}}				% zbiór elementów {a,b,c}
\newcommand{\pair}[1]{\left( #1 \right)}				% para elementów (a,b)
\newcommand{\Mod}[1]{\ \mathrm{mod\ #1}}				% lekko zmodyfikowane modulo
\newcommand{\conv}[1]{\equiv_{#1}}                      % przystawanie
\newcommand{\comp}[1]{\overline{ #1 }} 					% dopełnienie zbioru 
\newcommand{\annihilator}{\mathbf{E}}					% operator E
\newcommand{\seqAnnihilator}[1]{\annihilator \left\langle #1 \right\rangle} % E(a_n)
\newcommand{\sequence}[1]{\left\langle #1 \right\rangle} % <a_n>
\DeclareMathOperator{\lcm}{lcm}							% obsługa lcm w mathmode

% styl do kodu
\lstdefinestyle{code}{%
basicstyle=\ttfamily\small,
commentstyle=\color{green!60!black},
keywordstyle=\color{magenta},
stringstyle=\color{blue!50!red},
showstringspaces=false,
%numbers=left,
%numberstyle=\footnotesize\color{gray},
%numbersep=10pt,
tabsize=4,
rulecolor=\color{red},
breaklines=true
}

\newcommand{\code}[1]{\lstinline[style=code]{#1}} % kod inline
%%

\let\bar\overline

\include{subfiles/definitions}

\begin{document}
%\footnotesize
\small
\begin{multicols*}{5}
\section{Wariacje}
\subsection*{Liczba wariacji z powtórzeniami}
Dla zbiorów $A, B$ o odpowiednio $m, n$ elementach liczba funkcji ze zbioru $A$ w $B$
wynosi $n^m$, czyli $\abs{\{ f: A \to B \} = n^m}$.

\subsection*{Liczba wariacji bez powtórzeń}
Dla zbiorów $A, B$ o odpowiednio $m, n$ elementach liczba funkcji różnowartościowych
ze zbioru $A$ w $B$ wynosi $n (n - 1) \ldots (n - m + 1) = \frac{n!}{(n - m)!}$.

\subsection*{Liczba podzbiorów}
Zbiór $A$ o $n$ elementach ma $\abs{\{ B : B \subseteq A \} = 2^n}$ podzbiorów.

\subsection*{Para podzbiorów}
Dla $U$ będącego $n$-elementowym można wyznaczyć dwa jego podzbiory $A, B$ takie,
że $A \subseteq B$ na $\abs{\{ (A,B) : A \subseteq B \subseteq U \}} = 
\abs{\{ f: U \to \{ 0, 1, 2 \} \}} = 3^n$ sposobów.

\subsection*{Liczba permutacji}
Zbiór $U$ o $n$ elementach można spermutować na $n!$ sposobów.

\subsection*{Sufit, podłoga, część ułamkowa}
Niech $x \in \mathbb{R}, n \in \mathbb{Z}$, wtedy: \\
$ \floor{x} = n \Leftrightarrow n \leq x < n + 1 $ \\
$ \ceil{x}  = n \Leftrightarrow n - 1 < x \leq n $ \\
$ \fractional{x} = x - \floor{x} $

\subsection*{Własności sufitu i podłogi}
Niech $x \in \mathbb{R}, n \in \mathbb{Z}$, wtedy: \\
$ \floor{x + n} = n + \floor{x} $, ponieważ \\
$ \floor{x} + n \leq x + n < \floor{x} + n + 1 $. \\
Ponadto mamy: \\
$ \ceil{x + n} = n + \ceil{x} $ \\
$ \floor{-x} = -\ceil{x} $

\subsection*{Podzbiory k-elementowe}
Niech $\abs{U} = \{ 1, 2, \ldots, n \}$ oraz $P_n^k = {\{ A \subseteq U : \abs{A} = k \}}$.
Wtedy ${\frac{n!}{(n - k)!}} = {k! \abs{P_n^k}}$, czyli 
$\abs{P_n^k} = \frac{n!}{(n - k)! k!} = \binom{n}{k}$.

\subsection*{Symbol Newtona}
Dla $k, n \in \mathbb{N}$ takich, że $0 \leq k \leq n$ zachodzi:
$\binom{n}{k} = \binom{n}{n-k}$ \\
$\binom{n}{k} + \binom{n}{k+1} = \binom{n+1}{k+1}$

\subsection*{Kulki i szufladki}
$n$ kulek do $k$ szuflad można wrzucić na tyle sposobów, ile jest ciągów złożonych
z $n$ zer i $k-1$ jedynek, czyli $\binom{n+k-1}{k-1}$.

\subsection*{Dwumian Newtona}
Dla $n \in \mathbb{N}$ mamy 
$(x+y)^n = \sum\limits_{i=0}^{n} \binom{n}{i} x^i y^{n-i}$.

\subsection*{Zasada szufladkowa Dirichleta}
Niech $k, s \in \mathbb{N}_+$. Jeśli wrzucimy $k$ kulek do $s$ szuflad (Dirichleta),
a kulek jest więcej niż szuflad ($k > s$), to w którejś szufladzie będą przynajmniej
dwie kulki. Innymi słowy, dla skończonych zbiorów $A, B$, jeśli $\abs{A} > \abs{B}$,
to nie istnieje funkcja różnowartościowa z $A$ w $B$. Dla $k > s\cdot i$ kulek oraz
$s$ szuflad będzie w jakiejś szufladzie $i + 1$ kulek.
\section{Asymptotyka}
Niech $f, g : \mathbb{N} \to \mathbb{R} \geq 0$, wtedy możemy mówić o takich
funkcjach asymptotycznych:

\subsection*{Notacja dużego $O$}
Mamy ${f(n) = O(g(n))}$ wtw, gdy
${\exists (c > 0)} \ {\exists (n_0 \in \mathbb{N})} \ {\forall (n \geq n_0)} \
{f(n) < cg(n)}$. Ponadto dla ${C, a, \alpha, \beta \in \mathbb{R}}$ zachodzą takie
własności: \\
1. $\forall (\alpha, \beta) \ \alpha \leq \beta \Rightarrow n^\alpha = O(n^\beta)$, \\
2. $\forall (\alpha > 1) \ n^C = O(a^n)$, \\
3. $\forall (\alpha > 0) \ (\ln n)^C = O(n^\alpha)$. \\
Przydatna może okazać się reguła de l'Hospitala, więc gdy $f(n)$ i $g(n)$ dążą do
nieskończoności, to ${\lim\limits_{n \to \infty} \frac{f(n)}{g(n)}} = 
{\lim\limits_{n \to \infty} \frac{f^\prime (n)}{g^\prime (n)}}$.

\subsection*{Notacja małego $o$}
${f(n) = o(g(n))}$ wtw, gdy $\lim\limits_{n \to \infty} \frac{f(n)}{g(n)} = 0$.

\subsection*{Notacja duże Omega ($\Omega$)}
$f(n) = \Omega(g(n))$ wtw, gdy ${\exists (c > 0)} {\exists (n_0 \in \mathbb{N})} 
{\forall (n \geq n_0)}  {f(n) \geq cg(n)}$.

\subsection*{Notacja Theta ($\Theta$)}
$f(n) = \Theta(g(n))$ wtw, gdy ${f(n) = \Omega(g(n))} \wedge {f(n) = O(g(n))}$.

\subsection*{Notacja małe Omega ($\omega$)}
$f(n) = \omega(g(n))$ wtw, gdy $\lim\limits_{\mathclap{n \to \infty}} = \frac{f(n)}{g(n)} = \infty$.
\section{Arytmetyka modularna}

\subsection*{Funkcja modulo}
Niech $n, d \in \mathbb{Z}$ i $d \neq 0$. Wtedy: \\
$n \Mod{d} = n - \floor{\frac{n}{d}} \cdot d$. \\

$n \Mod{d} = r$ wtw, gdy ${0 \leq r < d} \ \wedge \ {\exists (k \in \mathbb{Z})} \ {n = kd + r}$

\subsection*{Przystawanie modulo}
$a \conv{n} b$ wtw, gdy $a \Mod{n} = b \Mod{n}$

\subsection*{Własności funkcji modulo}
1. $a + b \conv{n} a \Mod{n} + b \Mod{n}$ \\
2. $a \cdot b \conv{n} (a \Mod{n}) \cdot (b \Mod{n})$

\subsection*{Podzielność}
Niech $n, d \in \mathbb{Z}$ i $d \neq 0$. Wtedy: \\
1. $d | n$ wtw, gdy $\exists (k \in \mathbb{Z}) \ n = kd$ \\
2. $d | n$ wtw, gdy $n \Mod{d} = 0$ \\
3. $d | n$ wtw, gdy $n \conv{d} 0$ \\
4. $d | n_1 \wedge d | n_2$ to $d | (n_1 + n_2)$

\subsection*{Największy wspólny dzielnik (NWD, gcd)}
Niech $a, b \in \mathbb{N}$, wtedy \\
$\gcd(a,b) = \max \{ d \in \mathbb{N} : d|a \wedge d|b \}$

\subsection*{Własności NWD}
Dla $a > b$ względnie pierwszych ($a \perp b$) i $0 \leq m < n$: \\
$\gcd(a^n - b^n, a^m - b^m) = a^{\gcd(m,n)} - b^{\gcd(m,n)}$.

\subsection*{Algorytm Euklidesa}
Dla $a \geq b > 0$ korzystamy z własności:
$\gcd(a,b) = \gcd(b, a \Mod{b})$ oraz $\gcd(a, 0) = a$.
\begin{lstlisting}[style=code]
gcd(a, b):
    while b != 0:
        c = a mod b
        a = b
        b = c
    return a
\end{lstlisting}

\subsection*{Rozszerzony algorytm Euklidesa}
Dla $a \geq b > 0$: \\
$\exists (x, y \in \mathbb{Z}) \ xa + yb = \gcd(a, b)$
\begin{lstlisting}[style=code]
gcd(a, b):
    x = 1, y = 0, r = 0, s = 1
    while b != 0:
        c = a mod b
        q = a div b
        a = b
        b = c

        r' = r
        s' = s
        r = x - q * r
        s = y - q * s
        x = r'
        y = s'
    
    return a, x, y
\end{lstlisting}

\subsection*{Liczby względnie pierwsze}
Niech $a, b \in \mathbb{Z}$, wtedy te liczby są względnie pierwsze, gdy
$\gcd(a, b) = 1$.

\subsection*{Coś o liczbach pierwszych}
1. Jeśli $2^n - 1$ jest liczbą pierwszą, to $n$ jest liczbą pierwszą. \\
2. Jeśli $a^n - 1$ jest liczbą pierwszą, to $a = 2$. \\
3. Jeśli $2^n + 1$ jest liczbą pierwszą, to $n$ jest potęgą liczby $2$.
\section{Wzór włączeń i wyłączeń}

$\abs{\bigcup\limits_{i=1}^{n} A_i} = 
\sum\limits_{k=1}^{n} (-1)^{k-1}
\sum\limits_{I \subseteq \{ 1,\ldots,n \}} \abs{\bigcap\limits_{i \in I} A_i}$
\section{Rekurencja, zależności rekurencyjne}

\subsection*{Liczby Fibonacciego}
Niech $F_0 = 0, F_1 = 1$, wtedy $F_n = F_{n-1} + F_{n-2}$ dla $n > 1$.

\subsection*{Operator przesunięcia $\annihilator$}
Mamy ciąg $\sequence{a_n} = \sequence{a_0, a_1, \ldots, a_n, \ldots}$. Wtedy
$\seqAnnihilator{a_n} = \sequence{a_{n+1}} = \sequence{a_1, \ldots, a_n, \ldots}$.

\subsection*{Złożenie operatora przesunięcia}
$\annihilator^2 \sequence{a_n} = \annihilator \left( \seqAnnihilator{a_n} \right)
= \sequence{a_2, \ldots, a_n, \ldots}$

\subsection*{Operatory działające na ciągi}
$\sequence{a_n} + \sequence{b_n} = \sequence{a_n + b_n} = \sequence{a_0 + b_0, 
\ldots}$ \\
$c \sequence{a_n} = \sequence{c a_n} = \sequence{ca_0, ca_1, \ldots}$

\subsection*{Co anihiluje dane ciągi?}
$\sequence{\alpha}$ anihiluje $\annihilator - 1$. \\
$\sequence{\alpha a^i}$ anihiluje $\annihilator - a$. \\
$\sequence{\alpha a^i + \beta b^i}$ anihiluje $(\annihilator - a)(\annihilator - b)$. \\
$\sequence{\sum\limits_{k=0}^{n} \alpha_k a_k^i}$ anihiluje 
$\prod\limits_{k=0}^{n} (\annihilator - a_k)$. \\
$\sequence{\alpha i + \beta}$ anihiluje $(\annihilator - 1)^2$. \\
$\sequence{(\alpha i + \beta) a^i}$ anihiluje $(\annihilator - a)^2$. \\
$\sequence{(\alpha i + \beta) a_i + \gamma b^i}$ anihiluje
$(\annihilator - a)^2 (\annihilator - b)$. \\
$\sequence{\sum\limits_{k=0}^{n-1} \alpha_k i^k} a^i$ anihiluje 
$(\annihilator - a)^n$. \\

\subsection*{Dodatkowe własności anihilatorów}
Jeśli $\annihilator_A$ anihiluje $\sequence{a_i}$, to ten sam anihilator anihiluje
również ciąg $c \sequence{a_n}$ dla dowolnej stałej $c$. \\
Jeśli $\annihilator_A$ anihiluje $\sequence{a_i}$ i $\annihilator_B$ anihiluje
$\sequence{b_i}$, to $\annihilator_A \annihilator_B$ anihiluje
$\sequence{a_i} \pm \sequence{b_i}$.

\subsection*{Liczby Catalana}
$C_n$ oznacza $n$-tą liczbę Catalana, wyraża się przez 
$C_n = \sum\limits_{i=1}^{n} C_{i-1} C_{n-i}$ dla $C_0 = 0$. Można je również
przedstawić wzorami $C_n = \frac{1}{n+1} \binom{2n}{n} = \frac{(2n)!}{(n+1)!n!}$.
Spełniają one zależność $C_n = \binom{2n}{n} - \binom{2n}{n+1}$. \\
Liczby Catalana posiadają różne interpretacje kombinatoryczne, takie jak
liczba poprawnych rozmieszczeń nawiasów, liczba dróg w układzie współrzędnych
w I ćwiartce, liczba drzew binarnych, liczba podziałów wielokąta wypukłego na trójkąty.

\subsection*{Funkcje tworzące (OGF)}
Dla ciągu $\sequence{a_n}$ można utworzyć funkcję 
$\sum\limits_{i=0}^{\infty} a_i x^i = a_0 + a_1 x + a_2 x^2 + \ldots = A(x)$, która
jest funkcją tworzącą tego ciągu. Poniżej kilka typowych funkcji tworzących dla
ciągów: \\
$\frac{1}{1-x}$ dla ciągu $\sequence{1}$, czyli $\frac{n}{1-x}$ dla $\sequence{n}$. \\
$\frac{1}{1-2x}$ dla ciągu $\sequence{2^n}$. \\
$\frac{1}{(1-x)^2}$ dla ciągu $\sequence{1, 2, 3, \ldots}$. \\
$\frac{1}{1-x^2}$ dla ciągu $\sequence{0, 1, 0, 1, \ldots}$. \\

\subsection*{Przerwy pomiędzy wyrazami}
Funkcją tworzącą takiego ciągu $\sequence{a_0, 0, a_1, 0, a_2, 0, \ldots}$ jest
$\sum\limits_{i=0}^{\infty} a_i x^i = a_0 + a_1 x^2 + a_2 x^4 + \ldots = A(x^2)$.
Dla ciągu o wyrazach co $3$ miejsca byłoby to $A(x^3)$, dla $4$ to $A(x^4)$, dla
$n$ więc $A(x^n)$.

\subsection*{Co drugi wyraz ciągu (pochodne)}
Funkcją tworzącą $\sequence{a_0, 0, a_2, 0, a_4, 0, \ldots}$ jest 
$\frac{A(x) + A(-x)}{2}$, dla $\sequence{0, a_1, 0, a_3, \ldots}$ mamy
$\frac{A(x) - A(-x)}{2}$. \\ \\
Funkcją tworzącą takiego ciągu 
$\sequence{0, a_1, 2a_2, 3a_3, 4a_4, \ldots, ia_i, \ldots}$
jest pochodna funkcji $A(x)$ przesunięta o jedno miejsce w prawo, a więc $xA^\prime(x)$.

\subsection*{Wykorzystanie całek w OGF}
Aby odnaleźć funkcję tworzącą ciągu 
$\sequence{0, \frac{a_1}{1}, \frac{a_2}{2}, \ldots, \frac{a_i}{i}, \ldots}$ należy
scałkować funkcję tworzącą $A(x)$ i przesunąć ją w lewo:
$\int\limits_{0}^{1} \frac{A(x) - a_0}{x} dx = 
\sum\limits_{i=0}^{\infty} \frac{a_i}{i} x^{i}$.

\subsection*{Liczba podziałów liczby $n$}
Dowolne składniki: $\prod\limits_{i=1}^{\infty} \frac{1}{1 - x^i}$ \\
Różne składniki: $\prod\limits_{i=1}^{\infty} (1 + x^i)$ \\
Nieparzyste składniki: $\prod\limits_{i=1}^{\infty} (1 + x^{2i-1})$ \\
Składniki mniejsze od $m$: $\prod\limits_{i=1}^{m-1} \frac{1}{1 - x^i}$ \\
Różne potęgi $2$: $\prod\limits_{i=1}^{\infty} (1 + x^{2^i})$

\subsection*{Rekursja uniwersalna}
Niech $a, b, c$ będą dodatnimi stałymi, rozwiązaniem równania rekurencyjnego
$$
T(n) =
\begin{cases}
    b                    &\text{ dla } n = 1\\
    aT(\frac{n}{c}) + bn &\text{ dla } n > 1
\end{cases}
$$
dla $n$ będących potęgą liczby $c$ jest
$$
T(n) =
\begin{cases}
    O(n)                         &\text{ dla } a < c \\
    O(n \log n)                  &\text{ dla } a = c \\
    O\left( n^{\log_c a} \right) &\text{ dla } a > c
\end{cases}
$$
\end{multicols*}
\end{document}