\section{Asymptotyka}
Niech $f, g : \mathbb{N} \to \mathbb{R} \geq 0$, wtedy możemy mówić o takich
funkcjach asymptotycznych:

\subsection*{Notacja dużego $O$}
Mamy ${f(n) = O(g(n))}$ wtw, gdy
${\exists (c > 0)} \ {\exists (n_0 \in \mathbb{N})} \ {\forall (n \geq n_0)} \
{f(n) < cg(n)}$. Ponadto dla ${C, a, \alpha, \beta \in \mathbb{R}}$ zachodzą takie
własności: \\
1. $\forall (\alpha, \beta) \ \alpha \leq \beta \Rightarrow n^\alpha = O(n^\beta)$, \\
2. $\forall (\alpha > 1) \ n^C = O(a^n)$, \\
3. $\forall (\alpha > 0) \ (\ln n)^C = O(n^\alpha)$. \\
Przydatna może okazać się reguła de l'Hospitala, więc gdy $f(n)$ i $g(n)$ dążą do
nieskończoności, to ${\lim\limits_{n \to \infty} \frac{f(n)}{g(n)}} = 
{\lim\limits_{n \to \infty} \frac{f^\prime (n)}{g^\prime (n)}}$.

\subsection*{Notacja małego $o$}
${f(n) = o(g(n))}$ wtw, gdy $\lim\limits_{n \to \infty} \frac{f(n)}{g(n)} = 0$.

\subsection*{Notacja duże Omega ($\Omega$)}
$f(n) = \Omega(g(n))$ wtw, gdy ${\exists (c > 0)} {\exists (n_0 \in \mathbb{N})} 
{\forall (n \geq n_0)}  {f(n) \geq cg(n)}$.

\subsection*{Notacja Theta ($\Theta$)}
$f(n) = \Theta(g(n))$ wtw, gdy ${f(n) = \Omega(g(n))} \wedge {f(n) = O(g(n))}$.

\subsection*{Notacja małe Omega ($\omega$)}
$f(n) = \omega(g(n))$ wtw, gdy $\lim\limits_{\mathclap{n \to \infty}} = \frac{f(n)}{g(n)} = \infty$.