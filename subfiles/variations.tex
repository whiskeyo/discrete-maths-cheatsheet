\section{Wariacje}
\subsection*{Liczba wariacji z powtórzeniami}
Dla zbiorów $A, B$ o odpowiednio $m, n$ elementach liczba funkcji ze zbioru $A$ w $B$
wynosi $n^m$, czyli $\abs{\{ f: A \to B \} = n^m}$.

\subsection*{Liczba wariacji bez powtórzeń}
Dla zbiorów $A, B$ o odpowiednio $m, n$ elementach liczba funkcji różnowartościowych
ze zbioru $A$ w $B$ wynosi $n (n - 1) \ldots (n - m + 1) = \frac{n!}{(n - m)!}$.

\subsection*{Liczba podzbiorów}
Zbiór $A$ o $n$ elementach ma $\abs{\{ B : B \subseteq A \} = 2^n}$ podzbiorów.

\subsection*{Para podzbiorów}
Dla $U$ będącego $n$-elementowym można wyznaczyć dwa jego podzbiory $A, B$ takie,
że $A \subseteq B$ na $\abs{\{ (A,B) : A \subseteq B \subseteq U \}} = 
\abs{\{ f: U \to \{ 0, 1, 2 \} \}} = 3^n$ sposobów.

\subsection*{Liczba permutacji}
Zbiór $U$ o $n$ elementach można spermutować na $n!$ sposobów.

\subsection*{Sufit, podłoga, część ułamkowa}
Niech $x \in \mathbb{R}, n \in \mathbb{Z}$, wtedy: \\
$ \floor{x} = n \Leftrightarrow n \leq x < n + 1 $ \\
$ \ceil{x}  = n \Leftrightarrow n - 1 < x \leq n $ \\
$ \fractional{x} = x - \floor{x} $

\subsection*{Własności sufitu i podłogi}
Niech $x \in \mathbb{R}, n \in \mathbb{Z}$, wtedy: \\
$ \floor{x + n} = n + \floor{x} $, ponieważ \\
$ \floor{x} + n \leq x + n < \floor{x} + n + 1 $. \\
Ponadto mamy: \\
$ \ceil{x + n} = n + \ceil{x} $ \\
$ \floor{-x} = -\ceil{x} $

\subsection*{Podzbiory k-elementowe}
Niech $\abs{U} = \{ 1, 2, \ldots, n \}$ oraz $P_n^k = {\{ A \subseteq U : \abs{A} = k \}}$.
Wtedy ${\frac{n!}{(n - k)!}} = {k! \abs{P_n^k}}$, czyli 
$\abs{P_n^k} = \frac{n!}{(n - k)! k!} = \binom{n}{k}$.

\subsection*{Symbol Newtona}
Dla $k, n \in \mathbb{N}$ takich, że $0 \leq k \leq n$ zachodzi:
$\binom{n}{k} = \binom{n}{n-k}$ \\
$\binom{n}{k} + \binom{n}{k+1} = \binom{n+1}{k+1}$

\subsection*{Kulki i szufladki}
$n$ kulek do $k$ szuflad można wrzucić na tyle sposobów, ile jest ciągów złożonych
z $n$ zer i $k-1$ jedynek, czyli $\binom{n+k-1}{k-1}$.

\subsection*{Dwumian Newtona}
Dla $n \in \mathbb{N}$ mamy 
$(x+y)^n = \sum\limits_{i=0}^{n} \binom{n}{i} x^i y^{n-i}$.

\subsection*{Zasada szufladkowa Dirichleta}
Niech $k, s \in \mathbb{N}_+$. Jeśli wrzucimy $k$ kulek do $s$ szuflad (Dirichleta),
a kulek jest więcej niż szuflad ($k > s$), to w którejś szufladzie będą przynajmniej
dwie kulki. Innymi słowy, dla skończonych zbiorów $A, B$, jeśli $\abs{A} > \abs{B}$,
to nie istnieje funkcja różnowartościowa z $A$ w $B$. Dla $k > s\cdot i$ kulek oraz
$s$ szuflad będzie w jakiejś szufladzie $i + 1$ kulek.