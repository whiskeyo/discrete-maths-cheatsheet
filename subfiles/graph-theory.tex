\section{Teoria grafów}

\subsection*{Graf nieskierowany}
Graf nieskierowany to para zbiorów $(V, E)$, gdzie $E = \{ \{u, v\} : u, v \in V \}$.
$V$ to zbiór wierzchołków, $E$ to zbiór krawędzi.

\subsection*{"Patologie" w grafach}
Pętla to krawędź postaci $\{ v, v\}$, a krawędzie równoległe to dwie lub więcej
krawędzi łączących te same wierzchołki $u,v$ (dla $u \neq v$).

\subsection*{Graf prosty}
Graf $G = (V, E)$ jest prosty, jeśli nie zawiera pętli ani krawędzi równoległych.

\subsection*{Graf skierowany}
Graf nieskierowany to para zbiorów $(V, E)$, gdzie $E = \{ (u, v) : u, v \in V \}$.
$V$ to zbiór wierzchołków, $E$ to zbiór krawędzi skierowanych lub łuków.

\subsection*{Krawędź incydentna}
Krawędź $e$ jest incydentna do wierzchołka $u$, jeśli jeden z końców $e$ to $u$.

\subsection*{Stopień wierzchołka}
Stopień wierzchołka $u$ oznaczany przez $\deg(u)$ to liczba krawędzi incydentnych
do $u$. Każda pętla incydentna do $u$ dokłada się do stopnia $u$ liczbą $2$.

\subsection*{Lemat o uściskach dłoni}
Niech $G = (V, E)$ będzie nieskierowanym grafem. Wtedy
$\sum\limits_{v \in V} \deg(v) = 2 \abs{E}$.

\subsection*{Lemat o uściskach dłoni dla grafu skierowanego}
Niech $G = (V, E)$ będzie skierowanym grafem. Wtedy
$\sum\limits_{v \in V} \deg_\text{in}(v) = \deg_\text{out}(v)$.

\subsection*{Reprezentacje grafów}
Graf można reprezentować za pomocą list sąsiadów, macierzy sąsiedztwa lub macierzy
incydencji.

\subsection*{Izomorfizm grafów}
Dwa grafy nieskierowane proste $G = (V, E)$ i $H = (V^\prime, E^\prime)$ są
izomorficzne wtw, gdy istnieje bijekcja $f: V \to V^\prime$ taka, że
$\forall(u, v \in V) \ \{ u,v \} \in E \Leftrightarrow \{ f(u), f(v) \} \in E^\prime$.

\subsection*{Marszruta, ścieżka, droga, cykl}
1. Marszruta o długości $k$ to ciąg $\sequence{v_0, v_1, \ldots, v_k}$ taki, że \\
$\forall(0 \leq i < k) \ \{ v_i, v_{i+1} \} \in E$. \\
2. Droga to marszruta, w której żadna krawędź nie występuje dwukrotnie. \\
3. Ścieżka to marszruta, w której żaden wierzchołek nie występuje dwukrotnie. \\
4. Cykl to marszruta, w której pierwszy wierzchołek jest taki sam jak ostatni,
a poza tym, żaden wierzchołek nie występuje dwukrotnie.\\
$u-v$-marszruta to marszruta taka, że $v_0 = u$ i $v_k = v$, analogicznie 
definiujemy $u-v$-drogę i $u-v$-ścieżkę. \\
Marszruta/droga jest zamknięta, gdy $v_0 = v_k$. Zamknięta ścieżka to cykl.

\subsection*{Graf spójny}
Nieskierowany graf $G = (V, E)$ jest spójny, jeśli z każdego wierzchołka da się
dojść do innego, tzn. dla każdego wierzchołka $u,v \in V$ istnieje $u-v$-ścieżka.

\subsection*{Dopełnienie grafu}
Dopełnienie grafu $G$ oznaczamy przez $\comp{G}$, a definiujemy je jako graf
$\comp{G} = (V, E^\prime)$ taki, że $\{ u,v \} \in E^\prime$ wtw, gdy
$\{ u,v \} \notin E$.

\subsection*{Podgraf}
Podgrafem grafu $G = (V, E)$ jest dowolny graf $H = (V^\prime, E^\prime)$ taki,
że $V^\prime \subseteq V$ i $E^\prime \subseteq E$. Podgraf jest właściwy, jeśli
$G \neq H$.

\subsection*{Spójna składowa}
Spójna składowa grafu $G$ to dowolny podgraf spójny $H = (V^\prime, E^\prime)$
grafu $G$, który jest maksymalny ze względu na zawieranie, tzn. taki, że nie
istnieje podgraf spójny $H^\prime$, którego podgrafem właściwym jest $H$.

\subsection*{Drzewo i las}
Graf $G = (V, E)$ jest acykliczny, jeśli nie zawiera żadnego cyklu. Las jest
acyklicznym grafem, a drzewo acyklicznym grafem spójnym. Drzewa są spójnymi
składowymi lasu, a więc las składa się z drzew. \\
Drzewo jest najmniejszym grafem spójnym, a więc jeśli chcemy zbudować graf
spójny $G$ na zbiorze wierzchołków $V$, to $G$ musi być drzewem.

\subsection*{Liść}
Liść to wierzchołek o stopniu $1$. Dowolne drzewo o $n \geq 2$ wierzchołkach 
zawiera przynajmniej $2$ liście.

\subsection*{Most}
Most to krawędź, której usunięcie zwiększa liczbę spójnych składowych grafu,
ponadto żaden most nie leży na cyklu.

\subsection*{Charakteryzacja drzewa}
Niech $G = (V, E)$ będzie $n$-wierzchołkowym grafem nieskierowanym ($n \geq 1$).
Wtedy wszystkie następujące stwierdzenia są równoważne: \\
1. $G$ jest spójny i acykliczny ($G$ jest drzewem). \\
2. $G$ jest spójny i ma $n-1$ krawędzi. \\
3. $G$ jest acykliczny i ma $n-1$ krawędzi. \\
4. Dla każdego $u, v \in V$ w $G$ jest tylko jedna $u-v$-ścieżka. \\
5. $G$ jest spójny i każda krawędź jest mostem. \\
6. $G$ nie ma cykli, ale dołożenie jakiejkolwiek krawędzi tworzy cykl.

\subsection*{Liczba liści w dowolnym drzewie}
Niech $t_i$ oznacza liczbę wierzchołków stopnia $i$ w drzewie, wtedy \\
$t_1 = \sum\limits_{i=3}^{n} (i-2)t_i + 2$
oznacza liczbę liści w drzewie. Nie zależy ona od $t_2$, gdyż "przedłużenie"
liścia kolejną krawędzią nie zmienia liczby liści w drzewie.

\subsection*{Wierzchołek centralny, promień grafu}
Niech $d(u,v)$ oznacza odległość wierzchołków $u, v$, czyli długość najkrótszej
ścieżki łączącej je. Dla każdego wierzchołka $v$ definiujemy \\
$r(v) = \max \{ d(v, u) : u \in V(G) \}$. Wierzchołek $w$, dla którego 
$r(w) = \min \{ r(v) : v \in V(G) \}$ nazywamy wierzchołkiem centralnym grafu
$G$, a liczbę $r(G) = r(w)$ promieniem grafu $G$.

\subsection*{Graf dwudzielny}
Graf $G = (V, E)$ jest dwudzielny wtw, gdy istnieje podział zbioru $V$ na 
zbiory $A$ i $B$ taki, że dla każdej krawędzi $e \in E$ jeden koniec $e$
należy do zbioru $A$, a drugi do zbioru $B$. Podział wierzchołków nie zawsze
jest jednoznaczny! Graf $G$ jest dwudzielny wtw, gdy nie zawiera cyklu 
o nieparzystej długości.

\subsection*{Lemat o zamkniętej marszrucie}
Każda zamknięta marszruta o nieparzystej długości zawiera cykl o nieparzystej
długości.

\subsection*{Graf o minimalnym stopniu $k$}
Niech $G$ będzie grafem prostym, w którym każdy wierzchołek ma stopień 
przynajmniej $k$. Wówczas $G$ zawiera ścieżkę o długości $k$. Jeśli $k \geq 2$,
to $G$ zawiera cykl o długości przynajmniej $k+1$.

\subsection*{Algorytmy przeszukiwania grafów}
\textbf{Przeszukiwanie grafu w głąb}
\begin{lstlisting}[style=code]
DFS(u):
  u.visited = true
    for each neighbour v of u:
      if not v.visited
        DFS(v)
\end{lstlisting}

\textbf{Przeszukiwanie grafu wszerz}
\begin{lstlisting}[style=code]
BFS(v):
  queue Q = {}
  Q.enqueue(v)
  v.visited = true
    
  while (Q != empty):
    u = Q.dequeue()
      for each neighbour w of u:
        if not w.visited:
          Q.enqueue(w)
          w.visited = true
\end{lstlisting}
Czas działania DFS oraz BFS to $O(V + E)$.

\subsection*{Drzewo rozpinające}
Niech $G = (V, E)$ będzie grafem spójnym. Drzewo rozpinające grafu $G$ to 
podgraf $T = (V, E^\prime)$, który jest drzewem. $T$ zawiera wszystkie
wierzchołki grafu $G$.

\subsection*{Las rozpinający}
Niech $G = (V, E)$ będzie grafem niekoniecznie spójnym. Las rozpinający
grafu $G$ to podgraf $F = (V, E^\prime)$, który jest lasem, którego liczba
spójnych składowych jest taka sama jak liczba spójnych składowych grafu $G$.

\subsection*{Minimalne drzewo rozpinające (MST)}
Niech $G = (V, E)$ będzie grafem spójnym o nieujemnych wagach na krawędziach, a
graf $T = (V, E^\prime)$ jego drzewem rozpinającym. \\
Wagę definiuje funkcja $c : E \to \mathbb{R}_+$. Wtedy wagą drzewa rozpinającego
$c(T) = \sum\limits_{e \in E^\prime} c(e)$. Minimalnym drzewem rozpinającym (MST)
grafu $G$ jest drzewo rozpinające $T$ o minimalnej wadze.

\subsection*{Algorytmy na znajdowanie MST}
\textbf{Algorytm Kruskala} polega na dodawaniu kolejnych krawędzi w taki sposób,
aby nie stworzyły one żadnego cyklu.
\begin{lstlisting}[style=code]
KRUSKAL:
  sort(E) wzgledem wagi
  T = {}
  for i in [1, m]:
    if (T + {e(i)} nie tworzy 
        zadnego cyklu):
      T = T + {e(i)}
\end{lstlisting}

\textbf{Algorytm Prima} polega na dobieraniu najlżejszych krawędzi do grafu $T$.
\begin{lstlisting}[style=code]
PRIM:
  T = {}
  U = {1} (dowolny wierzcholek G)
  while (U != V):
    e = najlzejsza krawedz (u, v),
        taka ze u jest z U, 
        a v jest z V-U
    T = T + {(u, v)}  
    U = U + {v}
\end{lstlisting}

\textbf{Algorytm Boruvki} polega na dodawaniu najlżejszych krawędzi do $T$, łączeniu
ich w superwierzchołki i wykonywaniu algorytmu od początku.
\begin{lstlisting}[style=code]
BORUVKA:
  T = V
  while (T != MST):
    wybierz najmniejsza krawedz
    z najmniejsza waga i dodaj
    ja do zbioru E'

    gdy jest wiecej niz jedna
    spojna skladowa, polacz 
    wszystkie wierzcholki w
    superwierzcholki i wykonaj
    algorytm od poczatku
\end{lstlisting}
Wszystkie powyżej przedstawione algorytmy działają w czasie \\
$O(\abs{E} \cdot \log \abs{V})$.

\textbf{Algorytm Reverse-delete} polega na usuwaniu kolejnych krawędzi aby otrzymać $MST$.
\begin{lstlisting}[style=code]
KRUSKAL(edges[] E):
  sort(E) malejaco wzgledem wagi
  i = 0
  
  while i < size(E):
    edge = E[i]
    usun E[i]
    if graf nie jest spojny:
      E[i] = edge
      i = i + 1

  return edges[] E    
\end{lstlisting}
Złożoność czasowa tego algorytmu to $O(E \log V (\log \log V)^3)$.

\subsection*{Skojarzenie (matching)}
Niech $G = (V, E)$ będzie grafem spójnym. Skojarzenie grafu $G$ to dowolny pozdbiór
krawędzi $M \subseteq E$ taki, że żadne dwie krawędzie z $M$ nie mają wspólnego końca.

\subsection*{Największe skojarzenie}
Skojarzenie największe grafu $G$ to skojarzenie o maksymalnej liczbie krawędzi.

\subsection*{Wierzchołki skojarzone, wolne}
Niech $G = (V, E)$ będzie grafem spójnym, a $M$ jakimś skojarzeniem w $G$.
Wierzchołek $v \in V$ jest skojarzony w $M$, jeśli jest końcem jakiejś krawędzi
z $M$. Wierzchołek $v \in V$ jest wolny/nieskojarzony, jeśli żadna krawędź z $M$
nie jest z nim incydentna.

\subsection*{Ścieżka alternująca}
Ścieżka $P$ w grafie $G$ jest alternująca (względem $M$) jeśli krawędzie na $P$
na przemian należą i nie należą do $M$.

\subsection*{Ścieżka powiększająca}
Ścieżka $P$ w grafie $G$ jest powiększająca (względem $M$), jeśli jest 
alternująca względem $M$ i jej końce są nieskojarzone (w $M$).

\subsection*{Skojarzenie doskonałe/pełne}
Skojarzenie doskonałe/pełne grafu $G$ to skojarzenie, w którym każdy
wierzchołek z $V$ jest skojarzony.

\subsection*{Cykl alternujący}
Cykl $C$ w grafie $G$ jest alternujący względem $M$ jeśli krawędzie na $C$
na przemian należą i nie należą do $M$.

\subsection*{Twierdzenie Berge'a}
Skojarzenie $M$ grafu $G$ jest największe wtw, gdy $G$ nie zawiera ścieżki
powiększającej względem $M$.

\subsection*{Sąsiedztwo wierzchołków}
Niech $G = (V, E)$ będzie grafem a $W \subseteq V$ podzbiorem wierzchołków.
Sąsiedztwo $W$ oznaczane jako $N(W)$ definiujemy jako zbiór \\
$\{ v \in V : \exists(w \in W) \ \{ v, w\} \in E \}$.

\subsection*{Warunek Halla}
Niech graf $G = (A \cup B, E)$ będzie grafem dwudzielnym. \\
Dla każdego $A^\prime \subseteq A$ zachodzi 
$\abs{N(A^\prime)} \geq \abs{A^\prime}$ oraz dla każdego $B^\prime \subseteq B$ 
zachodzi $\abs{N(B^\prime)} \geq \abs{B^\prime}$.

\subsection*{Skojarzenie doskonałe w grafie dwudzielnym}
Graf dwudzielny $G$ zwiera skojarzenie doskonałe wtw, gdy spełniony jest 
w nim warunek Halla. 

\subsection*{Waga ścieżki, najlżejsza ścieżka}
Waga ścieżki $P$ to suma wag krawędzi leżących na $P$. Najlższejsza/najkrótsza
(względem $c : E \to \mathbb{R}_+$) ścieżka z $s$ do $t$ to ta, która ma 
najmniejszą wagę. \\
Niech $S \subseteq V$, a $s$ będzie ustalonym wierzchołkiem z $V$. Ścieżka $P$
z $s$ do $v$ jest prawie $S$-owa/osiągalna bezpośrednio z $S$, jeśli wszystkie
wierzchołki na $P$ oprócz $v$ są w $S$. \\
$d(v)$ to waga najkrótszej ścieżki z $s$ do $v$. \\
$t(v)$ to waga najkrótszej prawie $S$-owej ścieżki z $s$ do $v$, a gdy taka
ścieżka nie istnieje, to $t(v) = \infty$.

\subsection*{Algorytm Dijkstry}
Algorytm Dijkstry służy do znajdowania wagi najkrótszych ścieżek.
\begin{lstlisting}[style=code]
DIJKSTRA:
  S = {s}
  d(s) = 0

  for each neighbour v of s:
    t(v) = c(s, v)
  for other vertices:
    while (S != V):
      u = argmin{t(u): u not in S}
      S = S + {u}
      update all t(v):
        for each neighbour v 
        (not in S) of vertex u:
          t(v) = min{t(v),
                     d(u)+c(u,v)}
\end{lstlisting}

\subsection*{Pokrycie wierzchołkowe}
Niech $G = (V, E)$ będzie grafem. Pokrycie wierzchołkowe $G$ to dowolny
pozdbiór $V^\prime \subseteq V$ taki, że każda krawędź z $E$ ma przynajmniej
jeden z końców w $V^\prime$.

\subsection*{Najmniejsze pokrycie wierzchołkowe}
Najmniejsze pokrycie wierzchołkowe grafu $G$ to to spośród pokryć
wierzchołkowych $G$, które zawiera najmniej wierzchołków.

\subsection*{Pokrycie wierzchołkowe a skojarzenie}
Niech $G = (V, E)$ będzie grafem. Niech $M$ będzie jakimś skojarzeniem $G$, 
a $W$ jakimś pokryciem wierzchołkowym. Wtedy $\abs{M} \leq \abs{W}$.

\subsection*{Twierdzenie Koeniga}
Niech $G = (V, E)$ będzie grafem dwudzielnym, $M_{\max}$ największym skojarzeniem
$G$, a $W_{\min}$ najmniejszym pokryciem wierzchołkowym. Wtedy 
$\abs{M_{\max}} = \abs{W_{\min}}$.

\subsection*{Droga i cykl Eulera}
Niech $G = (V, E)$ będzie grafem spójnym nieskierowanym, niekoniecznie prostym.
Droga Eulera grafu $G$ to droga (krawędzie się nie powtarzają, wierzchołki mogą),
która zawiera każdą krawędź $e \in E$. Cykl Eulera grafu $G$ to droga zamknięta
(wierzchołek startowy jest taki sam jak końcowy), która zawiera każdą krawędź 
$e \in E$.

\subsection*{Warunki istnienia drogi/cyklu Eulera}
Spójny graf $G$ posiada drogę Eulera wtw, gdy zawiera $0$ lub $2$ wierzchołki
o stopniu nieparzystym. Spójny graf $G$ posiada cykl Eulera wtw, gdy wszystkie
jego wierzchołki mają stopień parzysty. W grafie skierowanym warunkiem na istnienie
cyklu Eulera jest taka sama liczba krawędzi wychodzących i wchodzących dla każdego
wierzchołka. \\

Aby spójny graf skierowany miał drogę Eulera, muszą zachodzić: \\
1. dla dokładnie jednego wierzchołka jest $\deg_\text{in}(v) = \deg_\text{out}(v) + 1$, \\
2. dla dokładnie jednego wierzchołka jest $\deg_\text{in}(v) + 1 = \deg_\text{out}(v)$, \\
3. dla każdego innego niż dwa powyższe wierzchołki jest 
$\deg_\text{in}(v) = \deg_\text{out}(v)$.

\subsection*{Ścieżka i cykl Hamiltona}
Niech $G = (V, E)$ będzie grafem spójnym nieskierowanym. Ścieżka Hamiltona grafu
$G$ to ścieżka (wierzchołki się nie powtarzają), która zawiera każdy wierzchołek
$v \in V$. Cykl Hamiltona grafu $G$ to cykl (wierzchołki się nie powtarzają),
który zawiera każdy wierzchołek $v \in V$.

\subsection*{Warunki istnienia drogi/cyklu Hamiltona}
Sprawdzenie, czy graf $G = (V, E)$ zawiera ścieżkę lub cykl Hamiltona jest 
problemem trudnym obliczeniowo - jest to problem NP-trudny.

\subsection*{Warunki konieczne na istnienie cyklu Hamiltona}
Jeśli graf $G = (A \cup B, E)$ jest dwudzielny, to warunkiem koniecznym na istnienie
cyklu Hamiltona jest $\abs{A} = \abs{B}$. \\ 
Jeśli graf $G = (V, E)$ zawiera cykl Hamiltona, to dla dowolnego zbioru $S \subseteq V$,
graf $G - S$ (powstały po usunięciu wierzchołków z $S$ wraz z incedyntnymi krawędziami)
zawiera co najmniej $\abs{S}$ spójnych składowych.

\subsection*{Twierdzenie Diraca}
Jeśli graf $G = (V, E)$ jest grafem prostym o co najmniej trzech wierzchołkach
i minimalnym stopniu wierzchołka $\delta (G) \geq \frac{\abs{V}}{2}$, to $G$ zawiera
cykl Hamiltona.

\subsection*{Twierdzenie Ore'a (albo Orego)}
Jeśli graf $G = (V, E)$ jest grafem prostym o co najmniej trzech wierzchołkach i takim,
że dla każdych dwóch wierzchołków $u$ i $v$ niepołączonych krawędzią zachodzi 
$\deg (u) + \deg (v) \geq \abs{V}$, to $G$ zawiera cykl Hamiltona.

\subsection*{Obliczanie najmniejszego wagowo cyklu Hamiltona}
Odległości między krawędziami zapisane są jako funkcja wagi $c: E \to R \geq 0$.
Zakładamy, że skrót zawsze się opłaca, czyli 
$\forall u, v, w \in V: \ c(u,v) \leq c(u,w) + c(w, v)$. Niech $OPT$ oznacza
sumaryczną długość optymalnej trasy. Wtedy $c(MST(G)) \leq OPT$. \\
Algorytm na obliczenie najmniejszego wagowo cyklu Hamiltona jest następujący: \\
1. Oblicz $MST(G)$. \\
2. Podwój $MST(G)$ otrzymując $T^2$. \\
3. Znajdź cykl Eulera $C_E$ podwojonego $MST(G)$, czyli $T^2$. \\
4. Skróć $C_E$ do cyklu Hamiltona. \\
Tak obliczona trasa ma wagę $\leq 2 OPT$.

\subsection*{Algorytm Christofidesa (najmniejszy wagowo cykl Hamiltona)}
1. Oblicz $MST(G)$. \\
2. Oblicz najmniejsze wagowo skojarzenie pełne $M$ na podgrafie zawierającym
wierzchołki $V^{-}$, które mają stopień nieparzysty w $MST(G)$. \\
3. Znajdź cykl Eulera $C_E$ multigrafu $MST(G) + M$. \\
4. Skróć $C_E$ do cyklu Hamiltona. \\
Tak otrzymana trasa ma wagę $\leq \frac{3}{2} OPT$.

\subsection*{Kolorowanie grafu}
Niech graf $G = (V, E)$ będzie grafem prostym. Kolorowaniem wierzchołkowym
grafu $G$ nazywamy funkcję $f: V \to \text{Kolory}$ taką, że 
$\forall (u, v) \in E: \ f(u) \neq f(v)$. Liczba chromatyczna grafu $G$ 
(oznaczana $\chi (G)$) to najmniejsza liczba kolorów, jaką można pokolorować 
graf $G$.

\subsection*{Własności liczby chromatycznej}
Przez $\omega (G)$ oznaczamy wielkość największej kliki zawartej w $G$.
Wtedy $\chi (G) \geq \omega (G)$. $\Delta (G)$ to największy stopień wierzchołka
w $G$. Wtedy $\chi (G) \leq \Delta (G) + 1$.

\subsection*{Twierdzenie Brooksa}
Jeśli $G$ nie jest kliką ani nieparzystym cyklem, to $\chi (G) \leq \Delta (G)$.

\subsection*{Algorytm sekwencyjny kolorowania grafu}
Niech $\text{Kolory} = \{ 1, 2, 3, \dots \}$ i $G = (V, E)$. Wtedy algorytmem
sekwencyjnym jest: \\
1. Ustaw wierzchołki z $V$ w pewien ciąg. \\
2. Dla każdego wierzchołka $v$ w kolejności dyktowanej przez ciąg wykonaj:
przypisz wierzchołkowi $v$ najmniejszą liczbę naturalną spośród takich, które
nie są przypisane żadnemu sąsiadowi $v$.

\subsection*{Graf planarny}
Graf $G$ jest planarny, gdy da się go narysować na płaszczyźnie w taki sposób,
by żadne dwie krawędzie się nie przecinały.

\subsection*{Rysunek grafu}
Łamana (linia wielokątna, linia łamana) to ciąg skończenie wielu odcinków,
z których każdy zaczyna się tam, gdzie kończy poprzedni; poza tym żadne dwa
odcinki nie mają punktów wspólnych. \\
Rysunek grafu $G = (V, E)$ na płaszczyźnie to funkcja różnowartościowa $f$ taka,
że odwzorowuje każdy wierzchołek $v \in V$ na punkt $f(v)$ płaszczyzny oraz każdą
krawędź $(u, v)$ na łamaną łączącą $f(u)$ z $f(v)$. \\
Mówimy, że rysunek nie ma przecięć, jeśli dla dowolnych dwóch krawędzi $e, e^\prime$,
$f(e) \cap f(e^\prime)$ może zawierać jedynie obrazy wspólnych końców $e$ i $e^\prime$. \\
Graf $G$ jest planarny, jeśli posiada rysunek bez przecięć. 

\subsection*{Graf płaski}
Konkretny rysunek bez przecięć grafu $G$ nazywamy grafem płaskim. \\
Ściana takiego grafu to spójny obszar płaszczyzny po usunięciu linii reprezentujących 
krawędzie. Innymi słowy, ściana to zbiór punktów płaszczyzny, które da się połączyć
krzywą nieprzecinającą żadnej krawędzi. \\
Granica ściany zawiera krawędzie styczne z tą ścianą. Długość granicy ściany to
długość zamkniętej marszruty przechodzącej przez wszystkie krawędzie granicy tej 
ściany. Niech $f_i$ oznacza długość granicy $i$-tej ściany grafu planarnego $G = (V, E)$,
a $l$ liczbę ścian $G$. Wtedy $\sum\limits_{i=1}^{l} f_i = 2 \abs{E}$.

\subsection*{Twierdzenie Jordana}
Zamknięta nieprzecinająca się łamana $C$ o skończonej liczbie odcinków dzieli
płaszczyznę na dokładnie dwie ściany, z których każda ma $C$ jako granicę.

\subsection*{Graf dualny}
Niech $G = (V, E)$ będzie grafem planarnym. Graf dualny $G*$ dla grafu płaskiego
$G$ tworzy się następująco: dla każdej ściany (włącznie ze ścianą zewnętrzną) grafu
$G$ dodajemy wierzchołek. Jeśli dwie ściany mają wspólną krawędź $e$, łączymy
wierzchołki utworzone w poprzednim kroku odpowiednie dla sąsiadujących ścian
krawędzią przecinającą tylko krawędź $e$. Graf dualny nie jest wyznaczony 
jednoznacznie (zależy od rysunku $G$).

\subsection*{Wzór Eulera}
Niech $G$ będzie spójnym grafem planarnym (niekoniecznie prostym) o $n$ wierzchołkach,
$m$ krawędziach i $f$ ścianach. Wówczas $n - m + f = 2$. Dla niespójnego grafu o $k$
spójnych składowych jest to wzór $n - m + f = k - 1$.

\subsection*{Liczba krawędzi grafu planarnego}
Niech $G$ będzie prostym grafem planarnym o $n \geq 3$ wierzchołkach. Wówczas liczba
krawędzi $m$ tego grafu nie przekracza $3n - 6$. Jeśli dodatkowo, $G$ nie zawiera
żadnego trójkąta, to $m \leq 2n - 4$.

\subsection*{Grafy homeomorficzne}
Grafy $G$ i $H$ są homeomorficzne, gdy jeden można przekształcić do drugiego za pomocą
skończonej liczby operacji następujących dwóch typów: \\
1. zamian krawędzi na ścieżkę o długości $2$, tj. w ten sposób dodajemy również
jeden nowy wierzchołek, \\
2. zamiana ścieżki $P = (u, v, w)$ takiej, że $v$ ma stopień $2$ na krawędź $(u, w)$,
jednocześnie usuwając $v$.

\subsection*{Twierdzenie Kuratowskiego}
Graf $G$ jest planarny wtedy i tylko wtedy, gdy nie zawiera podgrafu homeomorficznego
z $K_{3, 3}$ lub $K_5$.

\subsection*{Twierdzenie Heawooda}
Każdy graf planarny jest $5$-kolorowalny.

\subsection*{Sieć, przepływ w sieci}
Sieć to graf skierowany (digraf) $D = (V, E)$ z dwoma wyróżnionymi wierzchołkami
$s, t \in V$ zwanymi źródłem i ujściem i funkcją przepustowości $c: E \to R \geq 0$
na krawędziach. Niech $f: E \to R$, dla $v \in V$ definujemy 
$f^+(v) = \sum\limits_{e = (v, w): e \in E, w \in V} f(v, w)$ oraz
$f^-(v) = \sum\limits_{e = (w, v): e \in E, w \in V} f(w, v)$. \\ 
Funkcja $f$ jest przepływem, jeśli spełnia warunki przepustowości 
$\forall e \in E: \ 0 \leq f(e) \leq c(e)$ oraz jeśli spełnia warunek zachowania
przepływu: $\forall v \in V \{s, t\}: \ f^+(v) = f^-(v)$. Wartość przepływu $f$,
oznaczana jako $\abs{f}$ to $f^-(t) - f^+(t)$.

\subsection*{Ścieżka powiększająca}
Ścieżka powiększająca $P$ dla przepływu $f$ to ścieżka postaci
$(s = v_0, e_1, v_1, e_2, v_2, \dots, e_k, v_k, e_{k+1}, t = v_{k+1})$ taka,
że: \\
$\forall 0 \leq i \leq k: \ e_{i+1} \in E \wedge (e_{i+1} = (v_i, v_{i+1})
\vee e_{i+1} = (v_{i+1}, v_i)$, \\
$\forall 0 \leq i \leq k: \ e_{i+1} = (v_i, v_{i+1}) \Longrightarrow
f(e_{i+1}) < c(e_{i+1})$ (krawędź w przód), \\
$\forall 0 \leq i \leq k: \ e_{i+1} = (v_{i+1}, v_i) \Longrightarrow
f(e_{i+1}) > 0$ (krawędź w tył). \\
Luz ścieżki powiększającej $P$ to minimum z dwóch minimów: 
$\min\{ c(e) - f(e) \}$ po wszystkich krawędziach w przód ścieżki oraz
$\min\{ f(e) \}$ po wszystkich krawędziach wstecznych.

\subsection*{Zastosowanie ścieżki powiększającej}
Weźmy ścieżkę powiększającą $P$ taką jak powyżej dla przepływu $f$ o luzie $\epsilon$.
Zastosować $P$ do przepływu $f$ oznacza funkcję $f^\prime$ taką, że: \\
1. dla $e \in E \setminus P: f^\prime(e) = f(e)$, \\
2. dla $e \in P$ w przód: $f^\prime(e) = f(e) + \epsilon$, \\
3. dla $e \in P$ wstecznej: $f^\prime(e) = f(e) - \epsilon$. \\

Lemat: $f^\prime$ jest przepływem takim, że $\abs{f^\prime} = \abs{f^\prime} + \epsilon$.

\subsection*{Algorytm Forda-Fulkersona}
Niech $D = (V, E)$ będzie digrafem spójnym, $c: E \to R \geq 0$, $s, t \in V$ oraz
$\forall e \in E: \ f(e) \leftarrow 0$. Wtedy algorytm jest taki: \\
Dopóki istnieje ścieżka powiększająca $P$ dla $f$, wykonaj: \\
1. zastosuj $P$ do $f$, otrzymując $f^\prime$, \\
2. $f \leftarrow f^\prime$.

\subsection*{Algorytm znajdowania ścieżki powiększającej}
1. $R \leftarrow \{ s \}$. \\
2. Dopóki można, wykonuj: \\
2.1. Jeśli istnieje krawędź $e = (u, v): u \in R, v \notin R, f(e) < c(e)$,
     to dodaj $v$ do $R$. \\
2.2. Jeśli istnieje krawędź $e = (v, u): u \in R, v \notin R, f(e) > 0$,
     to dodaj $v$ do $R$. \\
Jeżeli $R$ zawiera $t$, to znaczy, że istnieje ścieżka powiększająca $P$ dla $f$.

\subsection*{Przepustowość przekroju}
$[S, T]$ to $s-t$ przekrój, jeśli $s \in S, t \in T, S \cup T = V, S \cap T = \emptyset$.
Przepustowość przekroju: \\
$c([S, T]) = \sum\limits_{e=(u,v) \in E: \ u \in S, v \in T} c(e)$. \\

Lemat: Niech $U \subset V$. Wtedy $f^+(U) - f^-(U) = 
\sum\limits_{v \in V} f^+(v) - f^-(v)$. Dla dowolnego $s-t$ przekroju $[S, T]$
zachodzi $\abs{f} \leq c([S, T])$.

\subsection*{Maksymalny przepływ, minimalne cięcie}
Twierdzenie: przepływ $f$ obliczony przez algorytm Forda-Fulkersona ma wartość
równą przepustowości pewnego $s-t$ przekroju. Zatem jest maksymalny. \\

Przepływ całkowitoliczbowy: jeśli przepustowość każdej krawędzi w sieci jest liczbą
całkowitą, to istnieje przepływ $f$ maksymalny, który jest całkowitoliczbowy.

\subsection*{Zastosowania przepływów}
Są to między innymi znajdowanie największego skojarzenia w grafach dwudzielnych, jak
i znajdowanie największego $b$-skojarzenia w grafach dwudzielnych. \\

Niech $b: V \in N$. Wtedy $M \subseteq E$ jest $b$-skojarzeniem, jeśli
$\forall v \in V: \ \deg_M (v) \leq b(v)$ (liczba krawędzi z $M$ incydentna do $v$
nie przekracza $b(v)$).