\section{Teoria grafów}

\subsection*{Graf nieskierowany}
Graf nieskierowany to para zbiorów $(V, E)$, gdzie $E = \{ \{u, v\} : u, v \in V \}$.
$V$ to zbiór wierzchołków, $E$ to zbiór krawędzi.

\subsection*{"Patologie" w grafach}
Pętla to krawędź postaci $\{ v, v\}$, a krawędzie równoległe to dwie lub więcej
krawędzi łączących te same wierzchołki $u,v$ (dla $u \neq v$).

\subsection*{Graf prosty}
Graf $G = (V, E)$ jest prosty, jeśli nie zawiera pętli ani krawędzi równoległych.

\subsection*{Graf skierowany}
Graf nieskierowany to para zbiorów $(V, E)$, gdzie $E = \{ (u, v) : u, v \in V \}$.
$V$ to zbiór wierzchołków, $E$ to zbiór krawędzi skierowanych lub łuków.

\subsection*{Krawędź incydentna}
Krawędź $e$ jest incydentna do wierzchołka $u$, jeśli jeden z końców $e$ to $u$.

\subsection*{Stopień wierzchołka}
Stopień wierzchołka $u$ oznaczany przez $\deg(u)$ to liczba krawędzi incydentnych
do $u$. Każda pętla incydentna do $u$ dokłada się do stopnia $u$ liczbą $2$.

\subsection*{Lemat o uściskach dłoni}
Niech $G = (V, E)$ będzie nieskierowanym grafem. Wtedy
$\sum\limits_{v \in V} \deg(v) = 2 \abs{E}$.

\subsection*{Reprezentacje grafów}
Graf można reprezentować za pomocą list sąsiadów, macierzy sąsiedztwa lub macierzy
incydencji.

\subsection*{Izomorfizm grafów}
Dwa grafy nieskierowane proste $G = (V, E)$ i $H = (V^\prime, E^\prime)$ są
izomorficzne wtw, gdy istnieje bijekcja $f: V \to V^\prime$ taka, że
$\forall(u, v \in V) \ \{ u,v \} \in E \Leftrightarrow \{ f(u), f(v) \} \in E^\prime$.

\subsection*{Marszruta, ścieżka, droga, cykl}
Marszruta o długości $k$ to ciąg $\sequence{v_0, v_1, \ldots, v_k}$ taki, że \\
$\forall(0 \leq i < k) \ \{ v_i, v_{i+1} \} \in E$. \\
Droga to marszruta, w której żadna krawędź nie występuje dwukrotnie. \\
Ścieżka to marszruta, w której żaden wierzchołek nie występuje dwukrotnie. \\
Cykl to marszruta, w której pierwszy wierzchołek jest taki sam jak ostatni,
a poza tym, żaden wierzchołek nie występuje dwukrotnie.\\
$u-v$-marszruta to marszruta taka, że $v_0 = u$ i $v_k = v$, analogicznie 
definiujemy $u-v$-drogę i $u-v$-ścieżkę. \\
Marszruta/droga jest zamknięta, gdy $v_0 = v_k$. Zamknięta ścieżka to cykl.

\subsection*{Graf spójny}
Nieskierowany graf $G = (V, E)$ jest spójny, jeśli z każdego wierzchołka da się
dojść do innego, tzn. dla każdego wierzchołka $u,v \in V$ istnieje $u-v$-ścieżka.

\subsection*{Podgraf}
Podgrafem grafu $G = (V, E)$ jest dowolny graf $H = (V^\prime, E^\prime)$ taki,
że $V^\prime \subseteq V$ i $E^\prime \subseteq E$. Podgraf jest właściwy, jeśli
$G \neq H$.

\subsection*{Spójna składowa}
Spójna składowa grafu $G$ to dowolny podgraf spójny $H = (V^\prime, E^\prime)$
grafu $G$, który jest maksymalny ze względu na zawieranie, tzn. taki, że nie
istnieje podgraf spójny $H^\prime$, którego podgrafem właściwym jest $H$.

\subsection*{Drzewo i las}
Graf $G = (V, E)$ jest acykliczny, jeśli nie zawiera żadnego cyklu. Las jest
acyklicznym grafem, a drzewo acyklicznym grafem spójnym. Drzewa są spójnymi
składowymi lasu, a więc las składa się z drzew. \\
Drzewo jest najmniejszym grafem spójnym, a więc jeśli chcemy zbudować graf
spójny $G$ na zbiorze wierzchołków $V$, to $G$ musi być drzewem.

\subsection*{Liść}
Liść to wierzchołek o stopniu $1$. Dowolne drzewo o $n \geq 2$ wierzchołkach 
zawiera przynajmniej $2$ liście.

\subsection*{Most}
Most to krawędź, której usunięcie zwiększa liczbę spójnych składowych grafu,
ponadto żaden most nie leży na cyklu.

\subsection*{Charakteryzacja drzewa}
Niech $G = (V, E)$ będzie $n$-wierzchołkowym grafem nieskierowanym ($n \geq 1$).
Wtedy wszystkie następujące stwierdzenia są równoważne: \\
1. $G$ jest spójny i acykliczny ($G$ jest drzewem). \\
2. $G$ jest spójny i ma $n-1$ krawędzi. \\
3. $G$ jest acykliczny i ma $n-1$ krawędzi. \\
4. Dla każdego $u, v \in V$ w $G$ jest tylko jedna $u-v$-ścieżka. \\
5. $G$ jest spójny i każda krawędź jest mostem. \\
6. $G$ nie ma cykli, ale dołożenie jakiejkolwiek krawędzi tworzy cykl.

\subsection*{Graf dwudzielny}
Graf $G = (V, E)$ jest dwudzielny wtw, gdy istnieje podział zbioru $V$ na 
zbiory $A$ i $B$ taki, że dla każdej krawędzi $e \in E$ jeden koniec $e$
należy do zbioru $A$, a drugi do zbioru $B$. Podział wierzchołków nie zawsze
jest jednoznaczny! \\
Graf $G$ jest dwudzielny wtw, gdy nie zawiera cyklu o nieparzystej długości.

\subsection*{Lemat o zamkniętej marszrucie}
Każda zamknięta marszruta o nieparzystej długości zawiera cykl o nieparzystej
długości.

\subsection*{Graf o minimalnym stopniu $k$}
Niech $G$ będzie grafem prostym, w którym każdy wierzchołek ma stopień 
przynajmniej $k$. Wówczas $G$ zawiera ścieżkę o długości $k$. Jeśli $k \geq 2$,
to $G$ zawiera cykl o długości przynajmniej $k+1$.

\subsection*{Drzewo rozpinające}
Niech $G = (V, E)$ będzie grafem spójnym. Drzewo rozpinające grafu $G$ to 
podgraf $T = (V, E^\prime)$, który jest drzewem. $T$ zawiera wszystkie
wierzchołki grafu $G$.

\subsection*{Las rozpinający}
Niech $G = (V, E)$ będzie grafem niekoniecznie spójnym. Las rozpinający
grafu $G$ to podgraf $F = (V, E^\prime)$, który jest lasem, którego liczba
spójnych składowych jest taka sama jak liczba spójnych składowych grafu $G$.

\subsection*{Minimalne drzewo rozpinające (MST)}
Niech $G = (V, E)$ będzie grafem spójnym o nieujemnych wagach na krawędziach, a
graf $T = (V, E^\prime)$ jego drzewem rozpinającym. \\
Wagę definiuje funkcja $c : E \to \mathbb{R}_+$. Wtedy wagą drzewa rozpinającego
$c(T) = \sum\limits_{e \in E^\prime} c(e)$. Minimalnym drzewem rozpinającym (MST)
grafu $G$ jest drzewo rozpinające $T$ o minimalnej wadze.

\subsection*{Algorytmy na znajdowanie MST}
\textbf{Algorytm Kruskala} polega na dodawaniu kolejnych krawędzi w taki sposób,
aby nie stworzyły one żadnego cyklu.
\begin{lstlisting}[style=code]
KRUSKAL:
sort(E) wzgledem wagi
T = {}
for i in [1, m]:
    if (T + {e(i)} nie tworzy 
        zadnego cyklu):
        T = T + {e(i)}
\end{lstlisting}

\textbf{Algorytm Prima} polega na dobieraniu najlżejszych krawędzi do grafu $T$.
\begin{lstlisting}[style=code]
PRIM:
T = {}
U = {1} (dowolny wierzcholek G)
while (U != V):
    e = najlzejsza krawedz (u, v),
        taka ze u jest z U, 
        a v jest z V-U
    T = T + {(u, v)}  
    U = U + {v}
\end{lstlisting}

\textbf{Algorytm Boruvki} polega na dodawaniu najlżejszych krawędzi do $T$, łączeniu
ich w superwierzchołki i wykonywaniu algorytmu od początku.
\begin{lstlisting}[style=code]
BORUVKA:
T = V
while (T != MST):
    wybierz najmniejsza krawedz
    z najmniejsza waga i dodaj
    ja do zbioru E'

    gdy jest wiecej niz jedna
    spojna skladowa, polacz 
    wszystkie wierzcholki w
    superwierzcholki i wykonaj
    algorytm od poczatku
\end{lstlisting}

\subsection*{Skojarzenie (matching)}
Niech $G = (V, E)$ będzie grafem spójnym. Skojarzenie grafu $G$ to dowolny pozdbiór
krawędzi $M \subseteq E$ taki, że żadne dwie krawędzie z $M$ nie mają wspólnego końca.

\subsection*{Największe skojarzenie}
Skojarzenie największe grafu $G$ to skojarzenie o maksymalnej liczbie krawędzi.

\subsection*{Wierzchołki skojarzone, wolne}
Niech $G = (V, E)$ będzie grafem spójnym, a $M$ jakimś skojarzeniem w $G$.
Wierzchołek $v \in V$ jest skojarzony w $M$, jeśli jest końcem jakiejś krawędzi
z $M$. Wierzchołek $v \in V$ jest wolny/nieskojarzony, jeśli żadna krawędź z $M$
nie jest z nim incydentna.

\subsection*{Ścieżka alternująca}
Ścieżka $P$ w grafie $G$ jest alternująca (względem $M$) jeśli krawędzie na $P$
na przemian należą i nie należą do $M$.

\subsection*{Ścieżka powiększająca}
Ścieżka $P$ w grafie $G$ jest powiększająca (względem $M$), jeśli jest 
alternująca względem $M$ i jej końce są nieskojarzone (w $M$).

\subsection*{Skojarzenie doskonałe/pełne}
Skojarzenie doskonałe/pełne grafu $G$ to skojarzenie, w którym każdy
wierzchołek z $V$ jest skojarzony.

\subsection*{Cykl alternujący}
Cykl $C$ w grafie $G$ jest alternujący względem $M$ jeśli krawędzie na $C$
na przemian należą i nie należą do $M$.

\subsection*{Twierdzenie Berge'a}
Skojarzenie $M$ grafu $G$ jest największe wtw, gdy $G$ nie zawiera ścieżki
powiększającej względem $M$.

\subsection*{Sąsiedztwo wierzchołków}
Niech $G = (V, E)$ będzie grafem a $W \subseteq V$ podzbiorem wierzchołków.
Sąsiedztwo $W$ oznaczane jako $N(W)$ definiujemy jako zbiór \\
$\{ v \in V : \exists(w \in W) \ \{ v, w\} \in E \}$.

\subsection*{Warunek Halla}
Niech graf $G = (A \cup B, E)$ będzie grafem dwudzielnym. \\
Dla każdego $A^\prime \subseteq A$ zachodzi 
$\abs{N(A^\prime)} \geq \abs{A^\prime}$ oraz dla każdego $B^\prime \subseteq B$ 
zachodzi $\abs{N(B^\prime)} \geq \abs{B^\prime}$.

\subsection*{Skojarzenie doskonałe w grafie dwudzielnym}
Graf dwudzielny $G$ zwiera skojarzenie doskonałe wtw, gdy spełniony jest 
w nim warunek Halla. 