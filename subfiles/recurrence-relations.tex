\section{Rekurencja, zależności rekurencyjne}

\subsection*{Liczby Fibonacciego}
Niech $F_0 = 0, F_1 = 1$, wtedy $F_n = F_{n-1} + F_{n-2}$ dla $n > 1$.

\subsection*{Operator przesunięcia $\annihilator$}
Mamy ciąg $\sequence{a_n} = \sequence{a_0, a_1, \ldots, a_n, \ldots}$. Wtedy
$\seqAnnihilator{a_n} = \sequence{a_{n+1}} = \sequence{a_1, \ldots, a_n, \ldots}$.

\subsection*{Złożenie operatora przesunięcia}
$\annihilator^2 \sequence{a_n} = \annihilator \left( \seqAnnihilator{a_n} \right)
= \sequence{a_2, \ldots, a_n, \ldots}$

\subsection*{Operatory działające na ciągi}
$\sequence{a_n} + \sequence{b_n} = \sequence{a_n + b_n} = \sequence{a_0 + b_0, 
\ldots}$ \\
$c \sequence{a_n} = \sequence{c a_n} = \sequence{ca_0, ca_1, \ldots}$

\subsection*{Co anihiluje dane ciągi?}
$\sequence{\alpha}$ anihiluje $\annihilator - 1$. \\
$\sequence{\alpha a^i}$ anihiluje $\annihilator - a$. \\
$\sequence{\alpha a^i + \beta b^i}$ anihiluje $(\annihilator - a)(\annihilator - b)$. \\
$\sequence{\sum\limits_{k=0}^{n} \alpha_k a_k^i}$ anihiluje 
$\prod\limits_{k=0}^{n} (\annihilator - a_k)$. \\
$\sequence{\alpha i + \beta}$ anihiluje $(\annihilator - 1)^2$. \\
$\sequence{(\alpha i + \beta) a^i}$ anihiluje $(\annihilator - a)^2$. \\
$\sequence{(\alpha i + \beta) a_i + \gamma b^i}$ anihiluje
$(\annihilator - a)^2 (\annihilator - b)$. \\
$\sequence{\sum\limits_{k=0}^{n-1} \alpha_k i^k} a^i$ anihiluje 
$(\annihilator - a)^n$. \\

\subsection*{Dodatkowe własności anihilatorów}
Jeśli $\annihilator_A$ anihiluje $\sequence{a_i}$, to ten sam anihilator anihiluje
również ciąg $c \sequence{a_n}$ dla dowolnej stałej $c$. \\
Jeśli $\annihilator_A$ anihiluje $\sequence{a_i}$ i $\annihilator_B$ anihiluje
$\sequence{b_i}$, to $\annihilator_A \annihilator_B$ anihiluje
$\sequence{a_i} \pm \sequence{b_i}$.

\subsection*{Liczby Catalana}
$C_n$ oznacza $n$-tą liczbę Catalana, wyraża się przez 
$C_n = \sum\limits_{i=1}^{n} C_{i-1} C_{n-i}$ dla $C_0 = 0$. Można je również
przedstawić wzorami $C_n = \frac{1}{n+1} \binom{2n}{n} = \frac{(2n)!}{(n+1)!n!}$.
Spełniają one zależność $C_n = \binom{2n}{n} - \binom{2n}{n+1}$. \\
Liczby Catalana posiadają różne interpretacje kombinatoryczne, takie jak
liczba poprawnych rozmieszczeń nawiasów, liczba dróg w układzie współrzędnych
w I ćwiartce, liczba drzew binarnych, liczba podziałów wielokąta wypukłego na trójkąty.

\subsection*{Funkcje tworzące (OGF)}
Dla ciągu $\sequence{a_n}$ można utworzyć funkcję 
$\sum\limits_{i=0}^{\infty} a_i x^i = a_0 + a_1 x + a_2 x^2 + \ldots = A(x)$, która
jest funkcją tworzącą tego ciągu. Poniżej kilka typowych funkcji tworzących dla
ciągów: \\
$\frac{1}{1-x}$ dla ciągu $\sequence{1}$, czyli $\frac{n}{1-x}$ dla $\sequence{n}$. \\
$\frac{1}{1-2x}$ dla ciągu $\sequence{2^n}$. \\
$\frac{1}{(1-x)^2}$ dla ciągu $\sequence{1, 2, 3, \ldots}$. \\
$\frac{1}{1-x^2}$ dla ciągu $\sequence{0, 1, 0, 1, \ldots}$. \\

\subsection*{Przerwy pomiędzy wyrazami}
Funkcją tworzącą takiego ciągu $\sequence{a_0, 0, a_1, 0, a_2, 0, \ldots}$ jest
$\sum\limits_{i=0}^{\infty} a_i x^i = a_0 + a_1 x^2 + a_2 x^4 + \ldots = A(x^2)$.
Dla ciągu o wyrazach co $3$ miejsca byłoby to $A(x^3)$, dla $4$ to $A(x^4)$, dla
$n$ więc $A(x^n)$.

\subsection*{Co drugi wyraz ciągu (pochodne)}
Funkcją tworzącą $\sequence{a_0, 0, a_2, 0, a_4, 0, \ldots}$ jest 
$\frac{A(x) + A(-x)}{2}$, dla $\sequence{0, a_1, 0, a_3, \ldots}$ mamy
$\frac{A(x) - A(-x)}{2}$. \\ \\
Funkcją tworzącą takiego ciągu 
$\sequence{0, a_1, 2a_2, 3a_3, 4a_4, \ldots, ia_i, \ldots}$
jest pochodna funkcji $A(x)$ przesunięta o jedno miejsce w prawo, a więc $xA^\prime(x)$.

\subsection*{Wykorzystanie całek w OGF}
Aby odnaleźć funkcję tworzącą ciągu 
$\sequence{0, \frac{a_1}{1}, \frac{a_2}{2}, \ldots, \frac{a_i}{i}, \ldots}$ należy
scałkować funkcję tworzącą $A(x)$ i przesunąć ją w lewo:
$\int\limits_{0}^{1} \frac{A(x) - a_0}{x} dx = 
\sum\limits_{i=0}^{\infty} \frac{a_i}{i} x^{i}$.

\subsection*{Liczba podziałów liczby $n$}
Dowolne składniki: $\prod\limits_{i=1}^{\infty} \frac{1}{1 - x^i}$ \\
Różne składniki: $\prod\limits_{i=1}^{\infty} (1 + x^i)$ \\
Nieparzyste składniki: $\prod\limits_{i=1}^{\infty} (1 + x^{2i-1})$ \\
Składniki mniejsze od $m$: $\prod\limits_{i=1}^{m-1} \frac{1}{1 - x^i}$ \\
Różne potęgi $2$: $\prod\limits_{i=1}^{\infty} (1 + x^{2^i})$

\subsection*{Rekursja uniwersalna}
Niech $a, b, c$ będą dodatnimi stałymi, rozwiązaniem równania rekurencyjnego
$$
T(n) =
\begin{cases}
    b                    &\text{ dla } n = 1\\
    aT(\frac{n}{c}) + bn &\text{ dla } n > 1
\end{cases}
$$
dla $n$ będących potęgą liczby $c$ jest
$$
T(n) =
\begin{cases}
    O(n)                         &\text{ dla } a < c \\
    O(n \log n)                  &\text{ dla } a = c \\
    O\left( n^{\log_c a} \right) &\text{ dla } a > c
\end{cases}
$$