\section{Arytmetyka modularna}

\subsection*{Funkcja modulo}
Niech $n, d \in \mathbb{Z}$ i $d \neq 0$. Wtedy: \\
$n \Mod{d} = n - \floor{\frac{n}{d}} \cdot d$. \\

$n \Mod{d} = r$ wtw, gdy ${0 \leq r < d} \ \wedge \ {\exists (k \in \mathbb{Z})} \ {n = kd + r}$

\subsection*{Przystawanie modulo}
$a \conv{n} b$ wtw, gdy $a \Mod{n} = b \Mod{n}$

\subsection*{Własności funkcji modulo}
1. $a + b \conv{n} a \Mod{n} + b \Mod{n}$ \\
2. $a \cdot b \conv{n} (a \Mod{n}) \cdot (b \Mod{n})$

\subsection*{Podzielność}
Niech $n, d \in \mathbb{Z}$ i $d \neq 0$. Wtedy: \\
1. $d | n$ wtw, gdy $\exists (k \in \mathbb{Z}) \ n = kd$ \\
2. $d | n$ wtw, gdy $n \Mod{d} = 0$ \\
3. $d | n$ wtw, gdy $n \conv{d} 0$ \\
4. $d | n_1 \wedge d | n_2$ to $d | (n_1 + n_2)$

\subsection*{Największy wspólny dzielnik (NWD, gcd)}
Niech $a, b \in \mathbb{N}$, wtedy \\
$\gcd(a,b) = \max \{ d \in \mathbb{N} : d|a \wedge d|b \}$

\subsection*{Własności NWD}
Dla $a > b$ względnie pierwszych ($a \perp b$) i $0 \leq m < n$: \\
$\gcd(a^n - b^n, a^m - b^m) = a^{\gcd(m,n)} - b^{\gcd(m,n)}$.

\subsection*{Algorytm Euklidesa}
Dla $a \geq b > 0$ korzystamy z własności:
$\gcd(a,b) = \gcd(b, a \Mod{b})$ oraz $\gcd(a, 0) = a$.
\begin{lstlisting}[style=code]
gcd(a, b):
    while b != 0:
        c = a mod b
        a = b
        b = c
    return a
\end{lstlisting}

\subsection*{Rozszerzony algorytm Euklidesa}
Dla $a \geq b > 0$: \\
$\exists (x, y \in \mathbb{Z}) \ xa + yb = \gcd(a, b)$
\begin{lstlisting}[style=code]
gcd(a, b):
    x = 1, y = 0, r = 0, s = 1
    while b != 0:
        c = a mod b
        q = a div b
        a = b
        b = c

        r' = r
        s' = s
        r = x - q * r
        s = y - q * s
        x = r'
        y = s'
    
    return a, x, y
\end{lstlisting}

\subsection*{Liczby względnie pierwsze}
Niech $a, b \in \mathbb{Z}$, wtedy te liczby są względnie pierwsze, gdy
$\gcd(a, b) = 1$.

\subsection*{Coś o liczbach pierwszych}
1. Jeśli $2^n - 1$ jest liczbą pierwszą, to $n$ jest liczbą pierwszą. \\
2. Jeśli $a^n - 1$ jest liczbą pierwszą, to $a = 2$. \\
3. Jeśli $2^n + 1$ jest liczbą pierwszą, to $n$ jest potęgą liczby $2$.